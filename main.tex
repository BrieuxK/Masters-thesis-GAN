\documentclass [12pt] {article}
\usepackage[english]{babel}
\usepackage[utf8]{inputenc}
\usepackage[T1]{fontenc}
\usepackage{amsfonts}
\usepackage{xcolor}
\usepackage{float}
\usepackage{xcolor}
\usepackage{layout} %plugin pour les marges
\usepackage{geometry} %pour changer les marges de la page
\geometry{left=2.5cm, right=2.5cm, top=3cm, bottom=3cm}
\usepackage{setspace} %interlignes
\usepackage{comment}  %ajouter des commentaires
\usepackage[pdftex]{graphicx} %insérer des images
\usepackage{amsmath}
\usepackage{tabularx} %tableaux
\usepackage{todonotes}
\usepackage{amsmath, amsfonts, amssymb}  %faire des systemes d'equations

\usepackage{tikz}
\usepackage{pgf-pie}

\usepackage{setspace} %interlignes
\usepackage{soul}     %soulignement
\usepackage{siunitx}  %symbole angstrom
\usepackage{hyperref} %Pour avoir des liens cliquables
\numberwithin{equation}{section} %Numéroter les équations par chapitre
\numberwithin{figure}{section}   %Numéroter les figures par chapitre
\usepackage{multicol} %adjoindre des images
\usepackage{cases} 
\usepackage{subcaption}
\usepackage{wrapfig}
\usepackage{csquotes}
\usepackage{biblatex}
\usepackage{caption}
\usepackage{floatrow}
\usepackage{geometry}
\usepackage[margin=1.5in]{geometry}
\usepackage{tikz}
\usetikzlibrary{patterns}
\usepackage{hyperref}
\hypersetup{
    colorlinks=true,
    linkcolor=black,
    filecolor=magenta,      
    urlcolor=black,
     citecolor=black
}
\urlstyle{same}
\pagestyle{headings}
\usepackage[style=numeric]{biblatex}
\title{Bibliography management: \texttt{biblatex} package}
\addbibresource{bibli.bib}
\setlength{\parindent}{0.5cm}
\setlength{\parskip}{1ex plus 0.5ex minus 0.2ex}
\newcommand{\hsp}{\hspace{20pt}}
\newcommand{\HRule}{\rule{\linewidth}{0.5mm}}
\newcommand{\Hquad}{\hspace{0.35em}}
\newcommand{\HHquad}{\hspace{0.2em}}
\captionsetup{position=below}
%%%%%%%%%%%%%%%%%%%%%%%%%%%%%%%%%%%%%%%%%%%%%%%%%%%%%%%%%%%%%%%%%%
\begin{document}
\begin{titlepage}
	\center
	\center 
		\textsc{\LARGE Université Catholique de Louvain}\\ [0,5cm]
		\textsc{\Large Faculté des sciences}\\[0.3cm]
		\textsc{\Large Ecole De Physique} \\[0.3cm]
		%\textsc{\large  Master I Physique: Finalité spécialisée en Physique Médicale}
		
		%\vfill
    %\vspace*{0.5 cm}
    
    \textsc{\LARGE LPHYS2336A : Neutrino Physics}\\[0.2 cm]
    
    %\textsc{\LARGE Detectors and Sensors}\\ [0.2 cm]
    
	\rule{\linewidth}{0.2 mm} \\[0.4 cm]
	\fbox{
	\begin{minipage}{0.8 \textwidth}
	\centering
	    	{\huge \bfseries Daya Bay Neutrino Experiment}\\
	¨%\rule{\linewidth}{0.2 mm} \\[1.0 cm]
	\end{minipage}

	}

     \begin{figure}[H]
        \centering
        \includegraphics[scale = 1]{The_Daya_Bay_Antineutrino_Detector_(8056998030).jpg}
    \end{figure}
	
	%\vfill
	\vspace{0.5 cm}
	
	
\begin{minipage}{0.4\textwidth}
	\begin{flushleft} \large
	\emph{Author}\\
    	Kaczmarczyk Brieux\\
		
		
	\end{flushleft}
\end{minipage}~
\begin{minipage}{0.4\textwidth}
	\begin{flushright} \large
		\emph{Teacher}\\ 
		   Lemaitre Vincent\\
		  
	\end{flushright}
\end{minipage}\\[2 cm]
  \includegraphics[scale = 0.15]{LogoUCL.png}\\[0.2cm]	% logo université
  	
           
            
	\vfill\vfill\vfill 	
	{\large\today}

\end{titlepage}

%%%%%%%%%%%%%%%%%%%%%%%%%%%%%%%%%%%%%%%%%%%%%%%%%%%%%%%%%%%%%%%%%%

\newpage
\tableofcontents

%%%%%%%%%%%%%%%%%%%%%%%%%%%%%%%%%%%%%%%%%%%%%%%%%%%%%%%%%%%%%%%%%%

\newpage

\section*{Acknowledgement}

Merci : Agni, Christophe, Florian, Oğuz.\\
Chloé\\
Kot Astro\\
Gwen\\

\newpage

\section*{Introduction}

The interest in studying Higgs pair production is motivated by two aspects. Firstly, this process, involving the coupling of the Higgs boson to itself ($H \rightarrow HH$), offers a unique way to probe the Higgs potential, potentially shedding light on questions relative to the fabric of our Universe. Secondly, Higgs pair production is an extremely rare phenomenon within the Standard Model, rendering it highly sensitive to potential new physics phenomena, especially at unprecedented energy scales. Additionally, the exploration of extended scalar sectors, as predicted by many models beyond the Standard Model, further underscores the significance of studying processes like Higgs pair production. However, this investigation encounters formidable challenges posed by overwhelming backgrounds from standard model processes.

All these experiments are being carried out inside particle accelerators and colliders. These are essential tools in the field of high-energy physics, providing researchers with the means to probe the fundamental constituents of matter and light the mysteries of the universe. Among these, the Large Hadron Collider (LHC) stands out as the most powerful particle accelerator ever built. It accelerates beams of protons to unprecedented energies and collides them head-on at several interaction points, where massive detectors such as the Compact Muon Solenoid (CMS) are positioned.

Nevertheless, the lifetimes of some of these particles are so short that they cannot even reach the first layer of our particle detectors. Hence, their decay products are our only tool to get informations of the presence of particles such as bosons. However, with the multitude of events happening inside a particle detector, the final state signature we are interested in could be faked in many different way. Thus, in high energy physics, distinguishing the background from the signal is a task of an extreme importance. At the moment, the strategy used to filter out this background suffers from different shortcomings. Resulting in a imperfect filtering of background events, and indeed, a mediocre isolation of the targeted final state signature. The goal of this project is to introduce a new and, potentially more effective, background modelling strategy in order to improve the performance of current analysis of the CMS data.

Unfortunately, such as task cannot be succesfully completed by mere human understanding. We will have to use a very powerful tool in order to process the humongous quantity of data we are facing : neural networks. To be more specific, deep neural networks will be the key to such a problem. These have already proved their worth in the field of high energy physics through modeling, signal identification and more. In this thesis, I will use a specific architecture of deep neural networks : the conditional Generative Adversarial Network (cGAN). This architecture is a variant of the standard GAN, where two deep neural networks are engaged in a competitive learning process. On one hand the generator learns to produce synthetic samples similar to the background distribution, while the discriminator learns to distinguish real data from synthetic ones. By training a cGAN on samples of background events, we aim to develop a sophisticated understanding of the underlying patterns and correlations, allowing us to generate fake but plausible samples reproducing the statistical properties of the background. In order to discriminate this background from the signal more efficiently in the future experiments.

In this thesis, we will start by a succinct explanation of the Standard Model, detailing both its remarkable successes and its recognized shortcomings. We will briefly explore how these deviations from the Standard Model motivate the needs of beyond Standard Model physics. One such theory is known as the \textit{2HDM} will be delved for its potential to address multiple of these shortcomings.\\
Then, we will explore one of the numerous available ways to obtain evidence supporting the proposed theory. We will list the common problems associated with other options and explain why the chosen option appears to be a great trade-off. The current state of research in this field will be discussed, highlighting why it might seem to be at a dead-end and how the approach discussed in this report might pave a new way forward.\\ 

Once this theoretical work about physics done, it will be time to dive into the realm of deep neural networks. We will break down the fundamental concepts, all the process leading to the final architecture of the network and finally a short summary of the former. Armed with this technical understanding, a detailed development of the network will be done. We will disucss the main obstacles encountered and how they were overcome. Then, the main results will be presented and discussed.\\

We will conclude this small chapter in the vast book of Physics by describing what this method has achieved so far, outlining its hard limitations, and exploring the different tools that could be used to address these challenges. By doing so, we aim to illustrate how this method could help others continue writing the forthcoming chapters in the ongoing story of physics.

%we will merge it with our knowledge of physics to carefully describe the background facing us, and how to select judiciously the right kinematic variables capable of efficiently characterizing the Drell-Yan background.\\
%In this thesis, we embark on a journey to harness the power of deep neural networks, specifically Generative Adversarial Networks (GANs), to model and control the background in Higgs pair production events. By training a GAN on samples of background events, we aim to develop a sophisticated understanding of the underlying patterns and correlations, allowing us to generate synthetic samples that faithfully reproduce the statistical properties of the background. This novel approach promises to enhance our ability to distinguish signal from background and unlock new avenues for discovery at the frontier of particle physics.


\newpage

\section{Two Higgs Doublet Model (2HDM)}

Since its formulation in the mid 70's, the Standard Model (SM) has achieved numerous accomplishements. From the beginning, it has been able to describe three of the fundamental forces being : the electromagnetic, weak and strong interactions ; aswell as classifying all known particles at that time. Since then, the evidence of other particles predicted by the SM such as : W/Z bosons (1983), top quark (1995), tau neutrino (2000) and the Higgs boson (2012) have added further confidence to this theory.\\

Despite all these succesful achievements, the SM has proven multiple times several shortcomings. The lack of gravitation in its formulation, translated by the incompatibility to reunite the SM with the most succesful theory about gravitation known : the General Relativity. To this day, several experiments have observed neutrinos oscillations, implying the existence of massive neutrinos; which is not part of the original SM. Same goes for the baryon asymmetry or for the existence of dark matter and dark energy. Thus, the necessity of new physics is obvious.\\

Despite these issues, the SM remains a very solid framework, exhibiting a wide range of phenomena, including spontaneous symmetry breaking, anomalies, and non-perturbative behavior. It can and should be used as a basis for building more exotic models that incorporate hypothetical particles, extra dimensions, and elaborate symmetries to explain experimental results at variance with the SM. Hence, the Standard Model will be the giant's shoulders on which we will stand.\\

The field of beyond-SM theories is broad with theories such as supersymmetry, String theories or loop quantum gravity which are notable attempts to build a \textit{Theory of everything}. However, we won't be as presumptuous in this work, we will focus on a simpler but yet, promising extension of the SM : the two Higgs Doublet Model (2HDM). As its name suggests, this model contains one more Higgs doublet than the SM, which leads to a richer phenomenology in the existence of five physical states of the Higgs boson : $H, h, A, H^\pm$. This model can be described by 6 parameters, four for the masses of the Higgs state : $m_H, m_h, m_A, m_{H^\pm}$, one for the ratio of the two vacuum expectation values : $\tan \beta$ and one for the mixing angle which diagonalizes the mass matrix of h and H : $\alpha$. Instead of only two for the SM : mass of the Higgs : $m_h$ and a single vacuum expectation value : $v$.\\

With all these additional parameters taken into account, the 2HDM would be able to explain several shortcomings of the SM, thus, allowing particle physics to perform a great leap forward in the understanding of our Universe. However, despite the mathematical credibility of this model, it lacks something primordial so far : an experimental verification of this theory. But before tackling this, let's delve into the theoretical frameworks of these models.

\subsection{The Standard Model}

\begin{figure}[H]
    \centering
    \includegraphics[scale = 1]{Standard_Model_table.png}
    \caption{The fundamental particles of the Standard Model}
    \label{prod}
\end{figure}

Before going into details about the 2HDM, it is important to mention the Standard Model (SM). We will simply go straight to the fundamentals of the SM by mentionning two crucial principles : first the extension of the gauge invariance principle as a local concept, and second the spontaneous symmetry breaking mechanism . The introduction of local gauge invariance generates the gauge bosons as well as the interactions of these gauge bosons with fermions, and also, if the gauge group is non abelian, among the gauge bosons themselves. The combination of local gauge invariance with the spontaneous symmetry breaking mechanism leads to the Higgs mechanism which generates the masses of weak vector bosons and fermions. Since the 2HDM is an extension of the symmetry breaking sector, in this chapter, we are going to review the mechanism of electroweak symmetry breaking and focus on the Higgs particle of the SM.

\subsubsection{Starting from electroweak interaction}

The Lagrangian of the electroweak interaction is :

\begin{equation}
    \mathcal{L}_{EW} =  - \frac{1}{4} W^{\mu \nu}_a W^a_{\mu \nu} - \frac{1}{4} B^{\mu \nu} B_{\mu \nu} + |D_\mu \Phi|^2 - V(\Phi),
\end{equation}
\begin{equation}
    V(\Phi) = \mu^2 + |\Phi|^2 + \lambda |\Phi|^4.
\end{equation}
where $\Phi$ is a complex scalar doublet with hypercharge $Y = +1$ under $SU(2)_L$ and $V(\Phi)$ is the most general, renormalisable potential.
The strength tensors are written as :
\begin{equation}
    W^a_{\mu \nu} = \partial_{\mu} W^a_{\nu} \HHquad - \HHquad \partial_{\nu} W^a_{\mu} \HHquad + \HHquad g \epsilon^{abc}W^b_{\mu} W^c_{\nu} 
\end{equation}
and
\begin{equation}
    B_{\mu \nu} = \partial_{\mu} B_{\nu} \HHquad + \HHquad \partial_{\nu} B_{\mu},
\end{equation}
$W^a_{\mu \nu}$ and $B_{\mu \nu}$ are the gauge fields of the symmetry group $SU(2)_L$ and $U(1)_Y$, respectively.\\
It is also important to note that, in order to obtain local gauge invariance, the derivatives need to be changed into covariant derivatives such as :
\begin{equation}
    D_{\mu} \HHquad = \HHquad \partial_{\mu} - igW^a_{\mu}T_a - ig'\frac{Y}{2}B_{\mu}
\end{equation}
where $T^{a} = \frac{\sigma^a}{2}$ with $\sigma$ being the Pauli matrices.\\
Two separate cases emerge here, depending on the sign of $\mu^2$. For $\mu^2 > 0$, the state of the lowest energy achievable corresponds to the annulement of the fields corresponding to :
\begin{equation}
   \langle \phi \rangle_0 &= \begin{pmatrix}
                        0 \\
                        0
                    \end{pmatrix} \\
\end{equation}
thus, there is no spontaneously symmetry breaking.\\
For $\mu^2 < 0$, , the symmetry is spontaneously broken, in this case the fundamental state is not unique anymore, it is actually a set of degenerated state of minimum energy corresponding to a circle : 
\begin{equation}
   \langle \phi \rangle_0 &= \frac{1}{\sqrt{2}} \begin{pmatrix}
                        0 \\
                        v
                    \end{pmatrix} \\
                    \quad \text{with} \Hquad v = \sqrt{\frac{-\mu^2}{\lambda}} \Hquad \text{(real)}
\end{equation}
with $v$ being the vacuum expectation value (vev), which will be introduced in more details in the following section.

\begin{figure}[H]
    \centering
    \includegraphics[scale = 0.5]{Higgs-potential-before-left-and-after-right-spontaneously-symmetry-breaking.png}
    \caption{Higgs potential without and with spontaneous symmetry breaking}
    \label{fig:enter-label}
\end{figure}

The doublet field $\phi$ can be expressed using the vev, the Higgs field and three Goldstone bosons $\phi_{1,2,3}$ :

\begin{equation}
    \Phi &= \frac{1}{\sqrt{2}} \begin{pmatrix}
                        \phi_1^+ + i\phi_2 \\
                        v \HHquad + \HHquad H \HHquad + \HHquad i \phi_3
                    \end{pmatrix}
\end{equation}
There are three massive vector bosons, which we will define as follows :
\begin{equation}
    W^{\pm}_{\mu} = \frac{1}{\sqrt{2}}( W^1_{\mu} \mp i W^2_{\mu}), \quad Z^0_{\mu} = \frac{1}{\sqrt{g^2 + g'^2}} (g W^3_{\mu} - g' B_{\mu})
\end{equation}
%The fourth vector field is orthogonal to $Z^0_{\mu}$, here is why it remains massless, it doesn't appear in the lagrangian :
The fourth vector field is orthogonal to $Z^0_{\mu}$, it doesn't appear in the lagrangian :
\begin{equation}
    A_{\mu} = \frac{1}{\sqrt{g^2 + g'^2}} (g W^3_{\mu} + g' B_{\mu})
\end{equation}
Mass terms are terms thar are bilinear in $W^{\pm}, Z, A$ :
\begin{equation}
    m_W = g \frac{v}{2}, \quad m_Z = \frac{v}{2} \sqrt{g^2 + g'^2}, \quad m_A = 0.
\end{equation}
Hence, the two massive gauge bosons are related via :
\begin{equation}
    \frac{m_W}{m_Z} = \frac{g}{\sqrt{g^2 + g'^2}} = \cos \theta_W
\end{equation}

In the case of non-abelian $SU(2)_L \times U(1)_Y$ electroweak theory, three of the gauge bosons require a mass : $W^{\pm}$ and $Z$, while the last gauge boson, the $\gamma$ stays massless knowing that the electric charged must be conserved through an exact symmetry.\\
By spontaneously breaking the symmetry $SU(2)_L \times U(1)_Y$ to $U(1)_{QED}$, three Goldstone bosons have been absorbed by the $W^{\pm}$ and Z bosons to form their longitudinal components and to get their masses. Since the $U(1)_Q$ symmetry is unbroken, the $\gamma$, which is associated to its generator, remains massless as it should be.\\

The physical bosons are, indeed, the photon A and the $W^\pm$ and Z bosons. In fact, $W^\pm$ bosons are mass eigenstates while $W^3_\mu$ and $B_\mu$ mix to give the two physical bosons $A_\mu$ and $Z_\mu$ :
\begin{equation}
    \begin{pmatrix}
                        Z^0_\mu \\
                        A_\mu
                    \end{pmatrix} \\
        &= \begin{pmatrix}
                        \cos( \theta_W) & -\sin(\theta_W)\\
                        \sin(\theta_W) & \cos(\theta_W)
                    \end{pmatrix} \\
    & \begin{pmatrix}
                        W^3_\mu \\
                        B_\mu
                    \end{pmatrix} \\
\end{equation}
with $m_A = 0$ and $m_W = m_Z \HHquad \cos \theta_W$, where $\theta_W$ is called the weak mixing angle and 
\begin{equation}
    \cos \HHquad \theta_W = \frac{g}{\sqrt{g^2 + g'^2}}, \quad \sin \HHquad \theta_W = \frac{g'}{\sqrt{g^2 + g'^2}}
\end{equation}
With the same doublet of scalar fields $\phi$, we can also generate the fermion masses. Indeed, we can add $SU(2)_L \times U(1)_Y$ gauge-invariant Yukawa interactions between  the scalar fields and the fermions which are $SU(2)$ doublets or singlets.\\

Thus, with the same isodoublet $\phi$ of scalar fields, we have generated the masses of both the weak vector bosons $W^\pm, Z$ and the fermions, while preserving the gauge symmetry in the lagrangian.

\subsection{The Higgs boson}

The Lagrangian of the Higgs field can be written as

\begin{equation}
    \mathcal{L} = \frac{1}{2} \partial_{\mu} H \partial{\mu} H - \lambda v^2 H^2 - \lambda v H^3 - \frac{\lambda}{4} H^4.
\end{equation}
With the mass of the Higgs boson being

\begin{equation}
    m^2_H = 2 \lambda v^2 = -2 \mu^2 \approx 125.3 \pm 0.4 \HHquad (stat.) \pm 0.5 \HHquad (syst.) \HHquad GeV/c^2
\end{equation}
where $\lambda$ is the Higgs self-coupling parameter.
From the previous lagrangian, several Higgs coupling can be derived using Feynman rules [\ref{uliege}] :
\begin{equation}
    g_{HHH} \HHquad = \HHquad 3 \frac{m^2_H}{v} \quad , \Hquad g_{HHHH} \HHquad = \HHquad 3 \frac{m^2_H}{v^2}.
\end{equation}
Altough Higgs couplings to fermions and bosons will be mentioned in a latter section, we can mention the couplings to these particles now. These couplings being :
\begin{equation}
    g_{Hf\Bar{f}} \HHquad = \HHquad  \frac{m_f}{v} \quad, \Hquad g_{HVV} \HHquad = \HHquad  -2 \frac{m^2_V}{v}  \quad, \Hquad g_{HHVV} \HHquad = \HHquad -2 \frac{m^2_H}{v^2} \Hquad ,
\end{equation}
with $v$ being the vacuum expectation value, with the accepted value of
\begin{equation}
    v \HHquad = \HHquad \sqrt{\frac{\mu^2}{\lambda}} \HHquad \approx \HHquad 246.22 \HHquad GeV [\ref{vev}].
    \label{vev}
\end{equation}
This value can be fixed in terms of the $W$ mass determined by the value of the Fermi constant $G_F$ :
\begin{equation}
    m_W = g\frac{v}{2} = \sqrt{\frac{\sqrt{2} g^2}{8 G_\mu}} %\left( \frac{\sqrt{2} g^2}{8 G_\mu}\right)^\frac{1}{2}
\end{equation}
This happens in muon decay, which occurs through gauge interactions mediated by W boson exchange, is a particular process through which $G_F$ is measured very accurately.

\begin{figure}[H]
    \centering
    \includegraphics[scale = 0.35]{muon_decay_Gf.png}
    \caption{Muon decay according to Fermi's theory of weak interaction. The coupling constant is $G_F$}
    \label{fig:enter-label}
\end{figure}

The Higgs couplings to fermions and bosons are predicted to be proportional to the corresponding particle masses, or squared-masses when it comes to boson masses. So, in Higgs production and decay processes, the dominant mechanisms involve the coupling of the Higgs boson to the heaviest particles available, in other words : $W^\pm$, $Z$ and the third generation of quarks and leptons.

\subsubsection*{Higgs decay channels}

Taking into consideration the previous section, it's no surprise to have the most dominant decays for Higgs boson to be $b \Bar{b}$ and $WW$. 
%At higher masses, the $WW$ channel remains the most dominant followed closely by $ZZ$ and from $350$ GeV, the $t \Bar{t}$ decay arises as the third most dominant. 

%\begin{figure}[H]
%    \centering
%    \includegraphics[scale = 0.6]{pie_chart.jpg}
%    \caption{Branching ratios of the Higgs boson decays.}
%    \label{higgs_decay_figure}
%\end{figure}

\begin{figure}[H]
    \centering
    \begin{tikzpicture}[scale = 0.8]
        \pie[rotate = 90, color = {red, blue, green, orange, pink}]{57/$b \Bar{b}$, 21/$W^- W^+$, 9/$gg$, 6/$\tau^- \tau^+$, 7/Other}
    \end{tikzpicture}
    \caption{Branching ratios of the Higgs boson decays.}
    \label{h decay}
\end{figure}

The category "Other" contains decays such as : $ZZ$, $cc$, $\gamma \gamma$, $\mu \mu$ and, theoretically, every massive particles since the Higgs boson couples to all of them.

\subsection{Flaws in the SM}

As mentioned earlier, the SM isn't an absolute model. There remain some points where it collides with experimental observations performed along the years. Let's go through some of them :

\subsubsection*{Massive neutrinos}

In the SM, neutrinos are considered as massless and only left-handed. However, it has been observed that these particles are, in fact, massive ! This has been shown through cosmological experiences [\ref{neutrino mass}] that have been able to determine an order of magnitude for neutrinos mass, being sub-eV, way lighter than other particles. Moreover, the concept of neutrino oscillations come from the mixing of the mass eigenstates and the flavour eigenstates. Hence, a neutrino of a specific flavour transitioning into a neutrino of another flavour during free propagation implies that, at least, one of these neutrinos must be massive. \\

\subsubsection*{Matter-antimatter (a)symmetry}

If the distribution of matter and antimatter was perfectly balanced, the current universe would be empty, each particle of matter would had been annilihated while interacting with a particle of antimatter.\\ However, since our Universe is not empty, it is definitely not the case ! In other words, there must have been an imbalance between matter and antimatter in the early Universe. A necessary condition to this excess of baryonic matter over antibaryonic one, is the baryon number violation. But C-symmetry violation is also needed so that the interactions which produce more baryons than anti-baryons will not be counterbalanced by interactions which produce more anti-baryons than baryons. CP-symmetry violation is similarly required because otherwise equal numbers of left-handed baryons and right-handed anti-baryons would be produced, as well as equal numbers of left-handed anti-baryons and right-handed baryons. Finally, the interactions must be out of thermal equilibrium, since otherwise CPT symmetry would assure compensation between processes increasing and decreasing the baryon number. The violation of baryon number, of C-symmetry, of CP-symmetry and interactions out of thermal equilibium are called the Sakharov conditions [\ref{Sakharov}].\\
However, the imbalance described by the SM is not large enough to correspond to our observations.

\subsubsection*{Dark Matter}

Thanks to cosmological observations, researchers have been able to highlight an inconsistence between the expected behaviour of our Galaxy compared to its actual behaviour, especially at great radiuses. Indeed, these regions seem to possess a greater energy density than what our telescopes tend to observe.\\
The possible explanation introduced to explain such phenomena is the introduction of a so-called \textit{Dark Matter} (DM), which would be a type of matter insensitive to electromagnetical interactions, electrically neutral, very long-lived (or completely stable) and massive at the same time. In other words, its only interactions would be with the Higgs boson and potentially through weak interactions (WIMPs [\ref{wimps}]). Moreover, it is know that this dark matter is cold [\ref{cold DM}], this means that it has a non-relativistic velocity distribution.\\
There is no sign of such a type of matter in the SM. However, if DM interacts weakly with the SM, it could be produced at the LHC experiments escaping the detector and leaving a large missing transverse momentum as its signature. [\ref{DM1}] [\ref{DM2}]


\subsubsection*{Hierarchy problem}

The hierarchy problem [\ref{Hierarchy}] tends to be the term used by physicists to describe a very important difference between the scale of mass of the electroweak bosons in one hand ($m_{W,Z,H} \approx 100 \HHquad GeV$[\ref{W mass}]) and on the other, the Planck mass ($m_{Planck} \approx 10^{19} \HHquad GeV $). In the SM, the mass term for the higgs boson can be written as :
\begin{equation}
    m^2 H^\dag H
\end{equation}
it is invariant under gauge and global symmetry on H, meaning that the higgs mass parameter can be modified by radiative corrections. Thus, the Higgs mass is modified by corrective terms from every scale with which it interacts, these terms being proportional to those scales. As mentionned earlier, those scales can go all the way up to the Planck mass, and so, the mass of the Higgs according to quantum field theory expectations is much (much) higher than the experimental result ($m_H \approx 125 \HHquad GeV$ [\ref{higgs mass}]). Currently, a "shaky" solution is the numerical cancellation of terms that results in the Higgs mass being reduced to its proper experimental values. However, relying on numerical cancellation is uncomfortable for many physicists.\\
One of the expected solution to this hierarchy problem is the use of SUper-SYmmetry (SUSY). Indeed, this theory involves the existence of plenty of other particles which could perform so-called \textit{miraculous cancellation} on the additionnal loops in the Higgs self-energy, solving this problem. Unfortunately, SUSY remains undiscovered as yet at the LHC and at all the other particle accelerator.


\subsection{The 2HDM}
The two-Higgs-Doublet Model (2HDM) is the most straightforward extension of the SM with one extra scalar doublet which contains more physical neutral and charged Higgs fields. Therefore, this model contains two complex doublets of scalar fields, $\phi_1$ and $\phi_2$:

\begin{equation}
    \phi_i &= \begin{pmatrix}
                        \phi_j^+ \\
                        \phi_j^0
                    \end{pmatrix} \\
    &= \begin{pmatrix}
                        \phi_1 + i \phi_2 \\
                        \phi_3 + i \phi_4
                    \end{pmatrix} \\
\end{equation}
with j = 1, 2. Thus, there are now eight degrees of freedom that will be used to give masses to the gauge bosons, instead of only four in the SM. In some cases, after symmetry breaking, three Goldstone bosons provide the longitudinal modes of the bosons $W^{\pm}$ and Z, that become massive. And there will remain five physical Higgs bosons : three neutral ones H, h, A and two charged ones $H^{\pm}$. Where H and h are scalar bosons such as $m_H > m_h$ and A is a pseudoscalar boson.\\
The most general scalar potential contains 14 parameters with CP-violating, CP-conserving and charge-violating minima. However, we can use several simplifying assumptions. First, we assume CP-conservation in Higgs sector, in order to draw a separation between scalars and pseudoscalars. Then, CP is not spontaneously broken. Lastly, discrete symmetries eliminate all quartic terms odd in either of the doublets from the potential, including a term wich softly breaks these symmetries. Under those assumptions, we can now formulate the most general scalar potential for two doublets $\Phi_1$ and $\Phi_2$ with $Y \HHquad = \HHquad +1$ as :

\begin{equation}
\begin{split}
    V &= m_{11}^2 \Phi^\dag_1 \Phi_1 + m_{22}^2 \Phi^\dag_2 \Phi_2 - m_{12}^2 (\Phi^\dag_1 \Phi_2 + \Phi^\dag_2 \Phi_1)+ \frac{\lambda_1}{2}(\Phi^\dag_1 \Phi_1)^2 + \frac{\lambda_2}{2}(\Phi^\dag_2 \Phi_1)^2 \\
    &\quad + \lambda_3 \Phi^\dag_1 \Phi_1 \Phi^\dag_2 \Phi_2 + \lambda_4 \Phi^\dag_1 \Phi_2 \Phi^\dag_2 \Phi_1 + \frac{\lambda_5}{2} \left[ (\Phi^\dag_1 \Phi_2)^2 + (\Phi^\dag_2 \Phi_1)^2 \right],
\end{split}
\end{equation}
where all the parameters are real.\\
As mentionned earlier, with two complex scalar SU(2) doublets there are eight fields :
\begin{equation}
    \Phi_a &= \begin{pmatrix}
                        \phi_a^+ \\
                        (v_a \HHquad + \HHquad \rho_a \HHquad + \HHquad i \eta_a)/\sqrt{2}
                    \end{pmatrix}, \quad a \HHquad = \HHquad 1,2.
\end{equation}
With $\phi^\pm$ corresponding to charged scalar bosons, $\eta$ to pseudoscalars and $\rho$ to scalars.\\
We can use these parameters to rewrite the scalars and pseudoscalar bosons as :
\begin{equation}
    A \HHquad = \HHquad \eta_1 \sin \beta \HHquad - \HHquad \eta_2 \cos \beta 
\end{equation}
\begin{equation}
    h \HHquad = \HHquad  \rho_1 \sin \alpha \HHquad - \HHquad \rho_2  \cos \alpha \quad and \quad H \HHquad = \HHquad -\rho_1 \cos \alpha \HHquad + \HHquad \rho_2  \sin \alpha
\end{equation}
Notice that the standard SM higgs boson would be :
\begin{equation}
\begin{aligned}
    H^{SM} \HHquad &= \HHquad \rho_1 \cos \beta \HHquad + \HHquad \rho_2 \sin \beta \\
    &= h \sin (\alpha - \beta) \HHquad - \HHquad H \cos(\alpha - \beta)
\end{aligned}
\end{equation}

\subsubsection*{The different types of 2HDM}

Two-Higgs-doublet models can introduce flavor-changing neutral currents (FCNC) which have not been observed so far and are considered as heavily suprressed by the GIM mechanism [\ref{FCNC}]. To avoid the prediction of such currents, we require that each group of fermions (up-type quarks : $Q = \frac{2}{3}$ : u,c,t; down-type quarks : $Q = \frac{-1}{3}$ : d,s,b and charged leptons) couples exactly to one of the two doublets $\phi$ as formulated here :
\begin{equation}
    \mathcal{L}_{Yukawa}^{2HDM} \HHquad = \HHquad - Y_d \Bar{Q}_L \Phi_d d_R \HHquad - \HHquad Y_d \Bar{Q}_L \Tilde{\Phi}_u u_R \HHquad - \HHquad Y_l \Bar{L}_L \Phi_l l_R \HHquad + \HHquad h.c.,
\end{equation}
with $\Phi_{d, u, l}$ corresponding to either $\phi_1$ or $\phi_2$.\\
By convention, up-type quarks always couple to $\phi_2$.

Depending on which type of fermions couples to which doublet 
$\phi$, one can divide two-Higgs-doublet models into the following classes [\ref{types of 2HDM}]: 
\begin{itemize}
    \item Type-I : all quarks and charged leptons couple to the same doublet : $\phi_2$
    \item Type-II : only up-type quarks couple to $\phi_2$, while down-type quarks and charged lepton couple to $\phi_1$
    \item Type X : all quarks couple to $\phi_2$, while charged leptons couple to $\phi_1$
    \item  Type Y : up-type quarks and charged leptons couple to $\phi_2$, while down-type quarks couple to $\phi_1$
\end{itemize}
The Type-II is the most studied case, since the couplings of the Minimal Super Symmetric Model (MSSM) are a subset of the couplings of Type-II 2HDM.\\
Another type, called Type-III [\ref{type 3}], exists. It relies on the inclusion of tree-level FCNC. Its goal is to be used for large energy scale (multi-TeV and higher), since, in that case, the previous solution used to exclude FCNCs seems unnatural. [\ref{type 3 bis}]

\subsubsection*{$\mathbb{Z}_2$ symmetry}

The most commonly used symmetry ensuring the absence of FCNCs is the $\mathbb{Z}_2$ symmetry. In the case of 2HDM, with two doublet $\Phi_1$, $\Phi_2$, a $\mathbb{Z}_2$ symmetry transforms the fields as :
\begin{equation}
    \mathbb{Z}_2 \Hquad : \Hquad \Phi_1 \rightarrow \Phi_1 \HHquad, \quad \Phi_2 \rightarrow -\Phi_2 \HHquad.
\end{equation}

\subsubsection*{Couplings to fermions}

To determine the Yukawa couplings, we can rewrite the Yukawa interaction term as :
\begin{equation}
\begin{split}
    \mathcal{L}_{\text{Yukawa}}^{\text{2HDM}} &= - \sum_{f = u,d,l} \left( \frac{m_f}{v} \xi^f_h \Bar{f} f h + \frac{m_f}{v} \xi^f_H \Bar{f} f H - i \frac{m_f}{v} \xi^{f}_A \right) \\
    &\quad - \left[ \frac{\sqrt{2}V_{ud}}{v} \Bar{u} (m_u \xi^u_A P_L + m_d \xi^d_A P_R) dH^+ + \frac{\sqrt{2}m_l}{v} \xi^l_A \nu_l l_R H^+ + \text{h.c.} \right]
\end{split}
\end{equation}
with $P_L$ and $P_R$ are the left and right projection operators, and the factors $\xi^f_H$, $\xi^f_h$, $\xi^f_A$ are parameters defined, in the case of Type-II 2HDM, as :
\begin{itemize}
    \item $\xi^u_H = \frac{sin \alpha}{sin \beta}$, $\xi^d_H =\frac{cos \alpha}{sin \beta}$, $\xi^l_H =\frac{cos \alpha}{sin \beta}$,
    \item $\xi^u_h = \frac{cos \alpha}{sin \beta}$, $\xi^d_h =\frac{-sin \alpha}{sin \beta}$, $\xi^l_h =\frac{-sin \alpha}{sin \beta}$,
    \item $\xi^u_A = cot \beta$, $\xi^d_A = tan \beta$, $\xi^l_A = tan \beta$.
\end{itemize}

\subsubsection*{2HDM and Super-symmetry}

Supersymmetry (SUSY) is a very elegant theory in particle physics which predicts the existence of so-called \textit{super-partners} or \textit{spartners} to each of the pre-existing particles in the SM. Each of these spartners would differ by a half-integer value from the spin of the SM particles, meaning that it's a symmetry transforming fermions to bosons and bosons to fermions. Thus, the SM needs to be extended by adding a new elementary particle for every known particle. Then, two point of view exist when it comes the SUSY, one involve a more simple formulation of the theory, with perfectly unbroken symmetry, meaning that particles and their correspondant spartners would have the same mass. On the other hand, there is also formulations about more complex symmetry, involving spontaneously broken symmetry allowing spartners to differ in mass.\\

SUSY would provide a very convenient solution to the hierarchy problem and would also be able to provide a DM candidate. Moreover, it unifies the three interactions at the grand unification theory scale.\\
The simplest supersymmetric extension to the SM is the Minimal Supersymmetric Standard Model (MSSM) [\ref{mssm}]. Knowing that this model requires a second Higgs doublet, this specific model of SUSY is directly included in the 2HDM.\\
However, no experimental proofs of SUSY as been found in high energy experiments.

\subsection{Di-Higgs physics}

\subsubsection*{Di-Higgs production}

\todo{prob cohérence entre $\lambda$ et $\kappa_\lambda$ }

The production of Higgs boson pairs inside particle colliders is very rare, a factor of thousand less likely than for the production of a single Higgs boson. The dominant process to produce a Higgs pair is the gluon-gluon fusion ($ggF$), where a Higgs pair can emerge as several types of final states, with probability given by branching decays of each boson shown at Fig. (\ref{h decay}). At leading order (LO), in other words, all vertices in the Feynman diagrams emerge from the lowest-order-Lagrangian, two separate cases appear, represented in the Feynman diagrams at Fig. (\ref{prod_ggf}). It's important to highlight that the box-like diagram is sensitive only to the Higgs-top quark coupling : $\kappa_t$, while the triangle one is sensitive $\kappa_t$ aswell as to the HHH coupling or Higgs self-coupling : $\kappa_\lambda$. There is a destructive interference between these two diagrams resulting in an overall small cross section of di-Higgs production. The total cross section for this process in the
SM is about $33.47$ fb. [\ref{33.47}]\\

Despite being the most dominant process, gluon-gluon fusion is not the only production channel. Indeed, the vector-boson fusion channel is able to produce pairs of Higgs bosons, despite a cross section being at least one order of magnitude smaller than $ggF$. This channel is still interesting to probe since it gives us access to several different couplings, such as the Higgs boson self-coupling, as in ggF, but also to new couplings like the HHVV coupling : $\kappa_{2V}$ or the HHV one : $\kappa_V$ with V being a W or Z boson.\\

It is also interesting to mention a third di-Higgs boson production channel being via $ggF$ with anomalous Higgs boson coupling. However, this channel won't be discussed in this work.

%The production of Higgs pairs at hadron colliders not only gives information on the Higgs sector but also on the BEH mechanism. It results in several types of final states, with probabilities given by the branching decays of each Higgs boson in the pair. One way of producing a Higgs boson pair is via Higgs self-coupling. Other interactions like the Higgs–fermion Yukawa interactions may also yield to the Higgs pair. Fig. (\ref{prod}) shows the Leading Order (LO) Feynman diagrams for the SM Higgs pair production. The LO means that all vertices in the Feynman diagrams emerge from the lowest-order-Lagrangian. All terms in the lowest-order-Lagrangian are at most of order $Q^2$ in the Taylor expansion of the typical momentum scale $Q$. Consequently, all vertices of LO Feynman diagrams contribute at order $Q^2$ in the Feynman matrix. The triangle diagram is sensitive to the Higgs self-coupling. The destructive interference of the box-like with the triangle diagram results in a small cross section for di-Higgs production. The prediction of the SM for the Higgs pair production cross section is $\sigma_{hh} = 33.49 fb$. The dominant process to produce a Higgs pair is gluon-gluon fusion. Although the cross section of the other pair production channels, like the vector-boson fusion channel, is at least one order of magnitude smaller, they are still interesting due to the different sensitivity to $\lambda$ or to new physics.


\begin{figure}[H]
    \centering
    \includegraphics[scale = 0.8]{DiHiggs_prod.png}
    \caption{Production channels for di-Higgs.}
    \label{prod_ggf}
\end{figure}

\begin{figure}[H]
    \centering
    \includegraphics[scale = 0.5]{DiHiggs_prod_VBF.png}
    \caption{Vector Boson Fusion (VBF) production channels for di-Higgs.}
    \label{prod_VBF}
\end{figure}

\subsection{Purpose of this project}

As stated earlier, the ability of discriminating signal from background is of crucial importance. This is the framework within which this project is being carried out.\\
The signal we are interested in is the channel $HH \rightarrow  b \Bar{b} \HHquad W^{+} W^-$, being the second largest decay branching fraction. We consider both $W$ bosons decaying to electrons or muons. The main backgrounds to this channel are the DY process, $t \Bar{t}$ production aswell as W + jets events.
We are going to use a conditional generative adversarial network trained on a sample of Drell-Yan events to generate background samples of the same process. Allowing future researches to have a better grasp of the charateristics of such a background, in order to filter it out efficiently.

\newpage

\input{Text/physics}

\newpage

\section{Neural networks}

In machine learning, a neural network (NN) is a model inspired by the neuronal organization found in the biological neural networks in animal brains.

A NN is made of connected units or nodes called artificial neurons, which loosely model the neurons in a brain. These are connected by edges, which model the synapses in a brain. An artificial neuron receives signals from the previous connected neurons, then processes them and sends a signal to following connected neurons. The "signal" is a real number, and the output of each neuron is computed by some non-linear function of the sum of its inputs, called the activation function. Neurons and edges typically have a weight that adjusts as learning proceeds. The weight increases or decreases the strength of the signal at a connection.\\

Typically, neurons are aggregated into layers. Different layers may perform different transformations on their inputs. Signals travel from the first layer (the input layer) to the last layer (the output layer), possibly passing through multiple intermediate layers (hidden layers). A network is called \textit{shallow} if it has few layers (3 or 4 in total) or \textit{deep} neural network if it has more than 3 or 4 total layers. In this project, we will focus on deep neural networks (DNN).\\

Artificial neural networks are used for predictive modeling, adaptive control, and other applications where they can be trained via a dataset. They are also used to solve problems in artificial intelligence. Networks can learn from experience, and can derive conclusions from a complex and seemingly unrelated set of information.

\begin{figure}[H]
    \centering
    \includegraphics[scale = 0.135]{Colored_neural_network.svg.png}
    \caption{Neural network with a single hidden layer}
\end{figure}

%Opposition de la méthode ABCD et du GAN. Le GAN serait capable de faire tout en un.\\
%Le GAN serait capable de faire le \textbf{morphing} de lui-même.\\

\subsection{Basic concepts of Neural Network}

\subsubsection*{Artificial neurons}

An artificial neuron is a mathematical model. In most cases, it computes the weighted average of its input and then possibly applies a bias to it, which won't be the case in this work. Afterwards, it passes this result through an activation function. This function is a nonlinear one that accepts a linear input and gives a nonlinear output.\\
In addition to the connection to other neurons and weights, a threshold can be implemented for every neuron. If the output of any individual node is above the specified threshold value, that node gets activated. In that case, it sends data to the next layer of the network, otherwise, it remains inactive and doesn’t transmit any data to the next layer of neurons. 
\begin{figure}[H]
    \centering
    \includegraphics[scale = 0.8]{ArtificialNeuronModel_english.png}
    \caption{Operation of an artificial neuron}
    \label{neuron}
\end{figure}


\subsubsection*{Backpropagation}

In machine learning, backpropagation refers to a method which computes the gradient of a loss function, i.e. a function representing the price paid for inaccuracy of predictions in classification problems, with respect to the weights of the network. It is computing the gradient one layer at a time, iterating backward from the last layer. It is then used to update the different parameters of the network.

\subsubsection*{Exploding and vanishing gradients}

The cases of vanishing and exploding gradients happen during the backpropagation when the slope of the activation function become progressively smaller or greater as we move backward through the layers of the NN. Obviously, this problem gets worse with DNN. The weight updates becomes either extremely small or extremely large, depending on the case, meaning that it will cause to completely stop the training process of the model.\\
This problem can be adressed by using specific activation functions, or using \textit{batch normalization} [\ref{batch}] that normalizes the inputs of each layers, reducing the risks of vanishing/exploding gradient.

\begin{figure}[H]
    \centering
    \includegraphics[scale = 1.1]{exploding_vanishing_grad.png}
    \caption{Gradients. Left : standard behaviour. Right : Vanishing and exploding gradient}
    \label{fig:enter-label}
\end{figure}

\subsubsection*{Under and over-fitting}

When using NNs for supervised learning, the procedure is usually to divide the data sample in 2 differents subsets : the training set (around $75\%$ of the initial set) and the testing set.\footnote{Note that a $3^{rd}$ set, the validation set, can also be used.} However, it can happen that the NN learns too well the training set, as in Fig.(\ref{overfit}). In this case, the model will perform very accurately on the training set, but once it will be tested on the other set, it will result in mediocre performances. In other words, the NN will learn "by-heart" the training set and won't be able to perform satisfyingly on a different sample, e.g. the testing one. This is called \textit{overfitting}. It represents an important challenge in the construction of a NN. One of the main causes of overfitting is a too complex model. Several tools can be used to avoid the overfitting of a model, we are going to discuss some later on.\\

On the order hand, there is underfitting, which represents a lack of training of the model leading to a too simplistic model unable to fit properly the data.

\begin{figure}[H]
    \centering
    \includegraphics[scale = 0.45]{underfitting_vs_overfitting.png}
    \caption{Left : underfitting. Middle : fitting. Right : overfitting.}
    \label{overfit}
\end{figure}

\subsection{Generative Adversarial Network (GAN)}

A GAN [\ref{GAN original}] is a specific class of machine learning framework used to approach generative modelling. Generative modelling is an unsupervised task, discovering and learning the patterns within the input data in order to generate new, fake but plausible, examples.\\
The network is divided in two sub-models, the generator and the discriminator, working in an adversarial way. The generator will create plausible examples based on an input sample of real data. The discriminator will determine whether the example provided is from the input sample (actual data) or if it is generated (fake data). Once the discriminator reaches a validity score of about $50\%$, the network is producing credible examples.

\begin{figure}[H]
    \centering
    \includegraphics[scale = 0.8]{GAN_scheme.png}
    \caption{Schematical representation of a GAN}
\end{figure}

The majority of GANs examples available are image-related GANs. The most popular applications are trained on databases such as MNIST [\ref{MNIST}], a set of handwritten digits or CIFAR-10 [\ref{CIFAR10}], which contain many differents datasets, one of them being a set of images of different vehicles such as cars, motorbikes, planes, ... However, in this work, images are not the desired output, we are rather aiming for numerical data. This represents a challenge in comparison to all the documentation available online. Beside the use of GANs in particle physics, application of numerical data GANs are also found in the sector of finances. [\ref{Finance}]\\
During the most part of this work, the input sample data used has been generated via \textit{MadGraph}. However, actual data can be used as input, leading to a more realistic training sample.\\

In the field of high energy physics, the use of GANs would solve the issue of limited simulated data samples by allowing large generated samples to be produced from much smaller datasets. However, these samples are not perfect, there are intrinsic mismodellings tied to these samples. Thus, it causes some of the largest sources of uncertainty in searches and measurements at the LHC.\\
In the approach followed in this work, this issue is resolved by \textit{blinding} the data in the signal region (SR) during the training session. It will falsely informs the GAN an absence of event in this region, leading to a generative model predicting a complete lack of background in SR. Afterwards, the prediction of the network can be interpolated or extrapolated into the SR.

\subsubsection*{Activation function}

The activation function of a node in NN is a function that calculates the output of the node based on its individual inputs and their weights. Nontrivial problems can be solved using only a few nodes if the activation function is nonlinear.\\
There is plenty of activation functions available, we will need to carefully choose the most appropriate to our goal. We can mention some popular functions, such as the sigmoid and the hyperbolic tangeant. They are both nonlinear, which is a crucial criterion in this case. However, the main drawback of these function are their limited sensitivity. Indeed, their nonlinear behaviour only stands in a short interval around 0, decreasing the sensitivity of the network for both large positive and large negative values. Moreover, these are relatively complex function to compute, due to the presence of exponentials in their mathematical formulation.

\begin{figure}[H]
    \centering
    \includegraphics[scale = 0.9]{sigmoid_vs_tanh.jpg}
    \caption{Activation functions. Left : sigmoid. Right : hyperbolic tangeant}
    \label{fig:enter-label}
\end{figure}

Now, we need a nonlinear activation function, with a sensitivity to large values and easy to compute. A function checking \textit{almost} all the boxes is the rectified linear unit (ReLU)[\ref{leaky relu}]. However, ReLU is not sensitive large negative value, as one can see on Fig. (\ref{relu and leaky relu}) which can cause issues. This leads us to our final choice : leaky rectified liner unit (Leaky ReLU). It stands apart from the standard ReLU thanks to the small gradient in the $]- \infty , 0]$ region. It allows the function to stay active for negative values, avoiding the case of a never activating neuron. Thus, it greatly improves the performance of the network despite adding another hyperparameter (let's call it $\alpha$) being the slope of the function in the negative region.\\

\begin{figure}[H]
    \centering
    \includegraphics[scale = 1.2]{ReLU-activation-function-vs-LeakyReLU-activation-function.png}
    \caption{Activation functions. Left : ReLU. Right : leaky ReLU}
    \label{relu and leaky relu}
\end{figure}

The leaky ReLU is not the only variation of the ReLU function, there is also GELU [\ref{GELU}], ELU [\ref{ELU}], SELU [\ref{SELU}]. Despite being quite recent (less than 10 years), these activation functions are widely used nowadays thanks to their numerous advantages.

\begin{figure}[H]
    \centering
    \includegraphics[scale = 0.4]{GELU_ELU_ReLU.png}
    \caption{GELU and ELU function in comparison to standard ReLU}
    \label{gelu elu}
\end{figure}

Despite all the advantages of leaky ReLU over the other activation functions, the former is not used for all the layers of the network. Indeed, the output layers of both the generator and discriminator are using the sigmoid function, as adviced in [\ref{Finance}]. One of the advantage is the easier computation of the results, since sigmoid in strictly bounded to $[0,1]$, which isn't the case of any ReLU variant.

\subsubsection*{Batch and batch size}
Batch size is a crucial component in deep learning training, it represents the number of samples (batches) used in one forward and backward pass through the network and has a direct impact on the accuracy and computational efficiency of the training process. Large batch sizes tends to lead to faster trainings but may result in lower accuracy and overfitting, while smaller batch sizes can provide better accuracy, but can be computationally expensive and time-consuming. The batch size can also affect the convergence of the model, meaning that it can influence the optimization process and the speed at which the model learns. Small batch sizes can be more susceptible to random fluctuations in the training data, while larger batch sizes are more resistant to these fluctuations but may converge more slowly.

\subsubsection*{Binary cross-entropy}

When developping a NN, we need a metric or a function describing the performance of our network, this will help us to optimize our model. If the predictions are close to the actual values, the loss function will be minimum, but if the predictions are far away from the actual values, it will be maximum. \\
There exist several different loss functions, the best choice depends on the problematic we are facing. In this case, we have a binary classifier (does the example has been generated or does it come from the actual data ?) so we choose a loss function called \textit{binary cross-entropy} [\ref{binary cross entropy}].\\
The standard cross-entropy is, in short, a tool to measure the difference between two distributions over the same set of events. With the \textit{entropy} being the number of bits required to transmit a randomly selected event from a probability distribution. For example, a skewed distribution has a low entropy while an equal probability distribution has a larger entropy. Another name for \textit{binary cross-entropy} is \textit{log loss}, in that expression it's easy to understand what makes this loss function interesting for us : the use of logarithms. Indeed, these will penalize heavily incorrect predictions. Here's how binary cross-entropy is computed :
\begin{itemize}
    \item If the label is 1, the cross-entropy is $-log(p)$ where p is the predicted probability
    \item but, if the label is 0, the cross-entropy is $-log(1 - p)$.
\end{itemize}
Then, the cross-entropy values are summed up for all examples. It can be written, under mathematical notations as : 
\begin{equation}
    \text{Log loss} = \frac{1}{N} \sum^N_{i=1} -\left( y_i \log(p_i) + (1-y_i) \log(1-p_i) \right),
\end{equation}
where $y_i$ represents the actual class, $p_i$ is the probability of class "1" and $1-p_i$ the probability of class "0".
%Additionally, it is a differentiable function, making it suitable for gradient-based optimization algorithms used in training neural networks.

\subsubsection*{Stochastic gradient descent optimizer}

Optimization algorithms are frequently used in machine learning to identify the best set of parameters that minimize the loss function.\\

In stochastic gradient descent (SGD) [\ref{SGD v9}], the algorithm quickly learns the direction of steepest descent using a single example of the training set at each time step. While this method has the distinct advantage of being fast, it may never converge to the global minimum. However, it approximates the global minimum closely enough. In practice, SGD is enhanced by gradually reducing the learning rate over time as the algorithm converges. In doing this, we can take advantage of large step sizes to go downhill more quickly and then slow down so as not to miss the global minimum. Due to its speed when dealing with humongous datasets, SGD is a popular choice.\\

It is useful to mention the existence of another popular optimizer : Adam. It has been proven that Adam outperforms SGD (with Nesterov momentum) on the MNIST data set [\ref{Adam}]. However, no performance gap between these two optimizers has been observed in this project.

\subsubsection*{Learning rate}

Learning rate (LR) is a common parameter of optimization algorithms that controls how big a step the gradient descent algorithm takes when tracing its path in the direction of steepest descent in the function space.

If the learning rate is too large, the algorithm takes a large step as it goes downhill. In doing so, gradient descent runs faster, but it has a high chance of missing the global minimum. Conversely, a too small learning rate makes the algorithm slow to converge (i.e., to reach the global minimum), but it is more likely to converge to the global minimum steadily. Empirically, examples of good learning rates are values in the range of 0.001, 0.01, and 0.1. In Fig.(\ref{LR}), with a good learning rate, the cost function $C(\theta)$ should decrease after every iteration.\\

\begin{figure}[H]
    \centering
    \includegraphics[scale = 0.45]{Gradient_descent.png}
    \caption{Learning rates. Left: Small learning rate. Right: Large learning rate.}
    \label{LR}
\end{figure}

In lot of low-level cases, the learning rate can be set at a constant value during the whole process of training. However, it is possible to adjust it dynamically during this process in order to reach better performance of the network. Doing so, two important obstacles can be tackled. In one hand, it allows the network to get out of local minima, and thus to converge to the global minimum with greater ease. On the other hand, as already proven [\ref{saddle}], saddle point are also critical points in optimizing paths. The gradients at saddle points tends to be very small and, thus, can slow the learning process. However, tweaking the learning rate allows the rapid traversal of saddle point plateaus.\\
This will be discussed in greater details in a future section.\\

\begin{figure}[H]
    \centering
    \includegraphics[scale = 0.5]{minmaxsaddle.png}
    \caption{Left : local minimum. Middle : local maximum. Right : saddle poiny}
    \label{fig:enter-label}
\end{figure}


\subsubsection*{Nesterov momentum}

One issue with SGD is that it can oscillate and take a long time to converge to a minimum, especially when the loss function has a complex structure or is highly non-convex. To mitigate that issue, we can add another parameter to the optimizer : the momentum.
Momentum is a technique that helps to mitigate this issue by adding a momentum term to the update rule.\\

The momentum term is essentially a weighted average of the past gradients, with the weighting decreasing exponentially as the gradients get further in the past. This helps to smooth out the oscillations, avoid local minima and accelerate convergence by allowing the optimizer to take larger steps in the direction of the minimum.\\

The Nesterov momentum [\ref{Nesterov}] is a variation aiming to improve the traditional momentum by making a subtle yet powerful change to the update rule. Instead of calculating the gradient at the current position, Nesterov’s Momentum calculates the gradient at a position slightly ahead in the direction of the accumulated momentum. This look-ahead step allows the optimizer to correct its course more responsively if it is heading towards a suboptimal direction.\footnote{For a indepth mathematical formulation, please check : [\ref{Nesterov}]}\\  
%It can be written, in mathematical notation, as : 
%\begin{itemize}
%    \item $change(t+1) = (momentum \times change(t)) - (step size \times f'(x(t)))$
%    \item $x(t+1) = x(t) + change(t+1)$
%\end{itemize}

\subsubsection*{L2 regularization}

Regularization is used in machine learning to avoid, as much as possible, overfitting. The idea is to add a penalty to the loss function as the model complexity increases such that the importance given to high order terms will decrease.

\begin{equation}
    J(\theta) = \frac{1}{2m} \left[ \sum^{m}_{i=1} (h_\theta (x^{(i)}) - y^{(i)})^2 + \textcolor{blue}{\lambda \sum^n_{j=1} \theta^2_j} \right], \quad \text{with the goal of minimizing } J(\theta).
    \label{regu}
\end{equation}

For L2 regularization in particular, the penalty term (written in blue) added to the loss function is the \textit{squared magnitude} of coefficient. It encourages smaller, more evenly distributed weights.


%\begin{figure}[H]
%    \centering
%    \includegraphics[scale = 0.8]{L2_regu.png}
%    \caption{Regularization in cost function}
%    \label{Regu}
%\end{figure}

\subsubsection*{Dropout layer}

A dropout layer is another type of regularization. The idea is to randomly ignore a subset of neurons of a specific layer during the training session, simulating training multiple neural network architectures to improve generalization.

\subsubsection*{He initialization}

In a NN, when weights are initialized randomly it can pose problem for the convergence of the network. To solve this problem, these weights have to be initialized in a specific way, that depends on the activation used in our model. In this case, we use an appropriate initialization for the (leaky) ReLU function : the \textit{He} weight initialization [\ref{He init}]. The idea behind this concept is to initialize weights in the following range : 
\begin{equation}
    N \left[ \left(-\frac{\sqrt{6}}{\sqrt{n_i(1+\alpha^2)}} \HHquad , \HHquad \frac{\sqrt{6}}{\sqrt{n_i(1+\alpha^2)}}  \right)\right]
\end{equation}
with $N$ a normal distribution, $n_i$ is the number of incoming network connections in the layer and $\alpha$ is the parameter of leaky ReLU function.\\
It's also useful to note that initializing weights can help furthermore mitigating exploding and vanishing gradients. 

\subsubsection*{Gradient clipping}

Gradient clipping [\ref{clipping}] adresses the problem of vanishing and/or exploding gradient by imposing a threshold on the gradients. If the gradients exceed this predefined threshold, they are rescaled to ensure they do not surpass the set limit. This rescaling step helps to keep the gradients within a manageable range, thus preventing drastic updates to the model’s parameters that might lead to instability or divergence during training.

\begin{figure}[H]
    \centering
    \includegraphics[scale = 0.55]{gradient_clipping.png}
    \caption{Gradient clipping. Left : without gradient clipping. Right : with gradient clipping.}
    \label{gradient clipping}
\end{figure}

\subsubsection*{Data preprocessing}

Due to the tools used for this project, standardization of datasets is a requirement for many machine learning estimators. Data might behave badly if the individual features do not more or less look like standard normally distributed data. In other words, Gaussian with zero mean and unit variance.\\
Several techniques exist, one of the most common one being the normalization/standardization of the input distributions in order to have a new distribution with a mean value of 0 ($\mu = 0$) and standard deviation of 1 ($\sigma = 1$). This process is done independently for each feature of the data. Given the distribution of the data, each value in the dataset will have the mean value subtracted, and then divided by the standard deviation of the whole dataset. In mathematical formulation, it represents : 
\begin{equation}
    z = \frac{x - \mu}{\sigma}
\end{equation}
\begin{equation}
    \text{with  } \mu = \frac{1}{N} \sum^N_{i=1} (x_i), \text{ and : } \sigma = \sqrt{\frac{1}{N} \sum^N_{i=1} (x_i - \mu)^2}.
\end{equation}

However, the method used in this work is a straightforward rescaling of the inputs. Indeed, since the activation functions used in this project are the sigmoid and leaky ReLU, it's convenient to rescale our data as [0,1], since it's the appropriate range for these functions to operate efficiently. Moreover, it helps to improve the stability of the network, which is crucial with a GAN. Neural networks in general are sensitive to the scale of input features, and having all the features within a similar range can prevent some of them to dominate the learning process.\\

Now let's consider a network with several hidden layers. As explained, the data preprocessing relates to the input of our network. From the perspective of the second layer, the output of the first layer simply corresponds to its inputs. Hence, it can be normalized. This concept is called \textit{batch normalization}. [\ref{batch norm}] However, it depends on a momentum parameter, adding yet another hyperparameter to the list.

\subsection{Adaptive learning rate techniques}

As stated in the dedicated section, the learning rate can be a complex hyperparameter to correctly tune. Indeed, both a too large and a too small learning rate will cause a poor training of our model. The most straightforward solution would be to determine judiciously a suitable LR for our model. Unfortunately, it's easier said than done.\\
A better idea is to have recourse to adaptive learning rates techniques. These methods will dynamically modify the value of LR during the training session, depending on the number of epochs already performed or on the performance of one or several specified metric(s).

\subsubsection{Learning rate scheduler}

This method tends to reduce the value of LR as the training session goes on, following a predefined schedule. The idea is to start with a large LR, in order to get closer quickly to the local minima. Then, the LR will be progressively decreased at each epoch in order to avoid overstepping the targeted minima.\\ 
However, the behaviour of this schedule has to be defined by the user, adding yet another pseudo-hyperparameter to the already long list of hyperparameters. Indeed, the scheduler can adopt a linear decrease strategy, exponential, time-based or even a step decay one.\\
Several variations of learning rate schedulers have been used for this project, but none of them have been retained due to a lack of performance improvement. [\ref{LR_S}]

\begin{figure}[H]
    \centering
    \includegraphics[scale = 0.25]{linear_step_exp_time_decay.png}
    \caption{Different scheduler behaviours}
    \label{lrs}
\end{figure}

\subsubsection{Cyclic learning rate}

As it name may suggests, this method implies the concept of cyclical variation of LR. This technique tackles another important concept beside the local minima : the saddle points. As stated in a previous section, saddle points may slow down the learning process. Moreover our network could stay stuck in a local minimum. To prevent these issues, an important LR is required, here's why the cyclic behaviour is interesting.
As for learning rate scheduler, there are several possible strategies, let's breakdown the three main ones.

\begin{figure}[H]
    \centering
    \begin{subfigure}{0.47\textwidth}
        \centering
        \includegraphics[scale = 0.45]{keras_clr_triangular.png}
        \caption{Cyclic triangular behaviour}
    \end{subfigure}
    \hspace{1.3cm}
    \begin{subfigure}{0.47\textwidth}
        \centering
        \includegraphics[scale = 0.45]{keras_clr_triangular2.png}
        \caption{Decreasing cyclic triangular behaviour}
        \label{t2}
    \end{subfigure}
    \hfill
    \begin{subfigure}{0.47\textwidth}
        \centering
        \includegraphics[scale = 0.45]{keras_clr_exp_range.png}
        \caption{Cyclic exponential behaviour}
        \label{exp}
    \end{subfigure}
    \caption{Different cyclical behaviours}
    \label{CLR types}
\end{figure}

The (\ref{t2}) and (\ref{exp}) cases mix the concept of "steadily" decreasing LR with cyclical LR. In this work, the decreasing triangular behaviour has been retained due to its better performance with the network.

However, this method isn't included in the \textit{keras.callbacks} package anymore. It could be found in the \textit{tensorflow.addons} package, unfortunately, it is deprecated since no major updates have been made recently. Thus, it may cause package clashes. To address this issue, the method had to be re-implemented with a few improvements.

\subsubsection{Reduce learning rate on plateau}

This technique is a scheduler variant. However, it is not based on the number of epochs performed but rather on the evolution (or absence of evolution) of one or more metric(s). The user needs to specify the metric to monitor as well as a \textit{patience} parameter, i.e. the number of epoch without any improvement of the metric, once this number exceeded a modification of the LR will be applied. Conversely to the other two methods, the change of LR is applied punctually, creating plateaux in the LR evolution, hence its name. [\ref{reduce_lr}]

\begin{figure}[H]
    \centering
    \includegraphics[scale = 0.3]{ReduceLROnPlateau.png}
    \caption{Behaviour of the reduce LR on plateau method for both LR and the monitored metric}
    \label{fig:enter-label}
\end{figure}
Despite being trialed, this strategy hasn't been retained.


\subsection{Conditional GAN}

A conditional GAN (cGAN) [\ref{cGAN}] is very similar to a standard GAN except it is able to conditionally generate samples based on an additional information provided to both the generator and the discriminator. This additional information is called the \textit{label}, allowing the network to return specific outputs. This level of control isn't available with standard GANs.\\
In the case of this work, this additional label will be the region of the sample, i.e. : signal region or control region.\\

\begin{figure}[H]
    \centering
    \includegraphics[scale = 0.85]{cGAN_scheme.png}
    \caption{Schematical representation of a cGAN}
\end{figure}

While training a standard GAN using blinded data it falsely informs the GAN that there are no events in the SR, leading to a generative model which predicts an absence of background events in the SR. Conversely, the cGAN learns the distribution of the background features conditioned on the blinding variable and so, despite being given no information about the background in the SR, can extrapolate its prediction into the SR. The cGAN can then be provided with the inclusive distribution of the blinding variable for all data events, and use what it learns in the unblinded data to interpolate the conditional generative model into the signal region, thereby predicting the values of the other variables.

\subsection{Sample}

For this project, \textit{MadGraph5} [\ref{MadGraph}] has been used to generate a sample of 20.000 MC Drell-Yan events, using the 2HDM extension for \textit{MadGraph5} at $13.6 TeV$. An important comment needs to be done here. In the case of 2HDM, there exist several process resulting in a pair of leptons from a proton-proton interaction, which is not the case in the SM. It is, then, mandatory to specify the mediator bosons expected, in our case : $Z$ and $\gamma$. Indeed, in the 2HDM, the three neutral higgs bosons can be other mediator bosons.

\begin{figure}[H]
    \centering
    \includegraphics[scale = 0.4]{6_var_distrib.png}
    \caption{Some relevant observables of the DY sample}
    \label{fig:enter-label}
\end{figure}

\subsection{Architecture}

Here is the composition of the cGAN model used for this work :
\begin{itemize}
    \item 5 layers for both discriminator and generator with 32 nodes each,
    \item SGD optimizer with momentum : 0.5 and gradient clipping limited at : 1.0,
    %\item L2 regularization with $\lambda = 0.01$,
    \item dimension of the latent vector set to 25,
    \item He initialization,
    \item leaky ReLU with $\alpha = 0.4$ for all layers except the last one of both networks, sigmoid otherwise,
    \item batch size set at 800 with batch normalization momentum at 0.8,
    \item the learning of the discriminator is set to \textit{False} when the generator is training,
    \item cyclic learning rate with decreasing triangular behaviour with maximum LR : 0.01, minimum LR : 0.0001, step size = 2000.
\end{itemize}

\subsection{Technical details}

As mentionned earlier, \textit{MadGraph5} has been used to generate a MC sample. The code has been written in \textit{Python}, using \textit{TensorFlow} [\ref{tf}] and \textit{Keras} [\ref{keras}] packages for the machine learning parts. The \textit{upROOT} [\ref{uproot}] package has been used for big data processing.

\subsection{Challenges}

Although capable of generating very accurate synthetic samples, GANs are also known to be hard to train [\ref{hard to train}] . Training two networks simultaneously means that when the parameters of one model are updated, the optimization problem changes. This creates a dynamic system that is harder to control. Non convergence is a common issue in GAN training. Deep models are usually trained using an optimization algorithm that looks for the lowest point of a loss function, but in a two-player-non-cooperative-game scenario, instead of reaching an equilibrium, the gradients may conflict and never converge, thus missing the global minimum. In other words, if the generator gets too good too fast, it may fool the discriminator and stop getting meaningful feedback, which in turn will make the generator train on bad feedback, leading to a collapse in output quality. An issue remains in the opposite case, even if these two networks are working in an adversarial way, one cannot outperform the other without compromising the performance of the GAN. These are widely known problems and several attempts were made to improve the stability of GANs. [\ref{OpenAI}]

At some point of this project, these issues were translated as such : the same network with the exact same set of hyperparameters can produce a totally different output distribution from one run to another, making it extremely difficult to fine-tune efficiently hyperparameters as no stable benchmark is available. Several methods were used to address this issue, as referenced previously. All of them improved the performance of the network. However, only the cyclical learning rate strategy was retained due to its better efficiency over other methods.\\ 
Plenty of other improving methods were proposed in this paper : [\ref{OpenAI}].\\

Moreover, a very effective way to adress the stability issue is by adapting our network to the desired goal. Indeed, there exists plenty of variations in GAN architecture, a non-exhaustive but important list can be found in [\ref{list_GAN}].

\newpage

\section{Statistical tests}

To compare efficiently two distributions, we cannot only rely on visual techniques as histograms. A more powerful tool is needed. In the following session, I will use the \textit{Kolmogorov-Smirnov test} (KS) [\ref{KS test}] and the \textit{Z-test} [\ref{Z test}] to assess whether or not two distributions are considered as sufficently similar. Let's briefly breakdown these two tests.

\subsubsection*{Kolmogorov-Smirnov}

This test has several applications. However, in this work, we will use its ability to test whether or not two underlying one-dimensional probability distributions differ. In this case, the Kolmogorov–Smirnov statistic is :

\begin{equation}
\begin{aligned}
    D_{n,m} &= \sup |F_{1,n}(x) - F_{2,m}(x)|, \\
    \text{with } F_{a,b} &= \frac{\text{number of elements in the sample $\leq t$}}{n} = \frac{1}{n} \sum^n_{i=1} \textbf{1}_{X_i \le t},
\end{aligned}
\end{equation}
with $n$, $m$ the number of events in the distribution, $t$ a fixed parameter and $F_{a,b}$ an empirical distribution function (commonly also called an empirical cumulative distribution function) which can be expressed using a \textit{Bernouilli random variable} : $\textbf{1}_{X_i \le t}$. From there, we can compute the p-value, the probability of obtaining test results at least as extreme as the result actually observed. If this value is \textbf{greater} than a specified threshold (for instance, $0.05$), we conclude a significant association between the two populations.
In addition to the p-value, the \textit{scipy} function used also returns the KS statistic. It represents the maximum distance between the two empirical cumulative distribution functions of the two samples.

\begin{figure}[H]
    \centering
    \includegraphics[scale = 1]{KS statistic.png}
    \caption{KS statistic (in black) between two empirical cumulative distribution functions (red and blue).}
    \label{fig:enter-label}
\end{figure}

%\subsubsection*{Z-test}

%The Z-test is a statistical test used to determine whether or not two population means are different given the standard deviation and a large sample size. The approach used is :

%\begin{equation}
%    Z = \frac{\Bar{x}_1 - \Bar{x}_2}{\sqrt{\frac{\sigma_1^2}{n_1} + \frac{\sigma_2^2}{n_2}}},
%\end{equation}
%with $\Bar{x}_i$ the mean of sample \textit{i}, $\sigma_i$ the standard deviation of sample \textit{i} and $n_i$ the size of sample \textit{i}.
%The value of $Z$ can be positive or negative. However, only its norm is interesting. In this work we have considered three cases :
%\begin{enumerate}
%    \item $Z < 2$ : the samples are very similar
%    \item $Z > 3$ : the samples are different
%    \item $ 3 \geq Z \geq 2$ : the samples are vaguely similar
%\end{enumerate}

%\subsubsection*{Wilcoxon signed test}

%The Wilcoxon signed test [\ref{wilcoxon}] is a non-paramteric test with several applications. However, we will use its ability to test whether or not two underlying one-dimensional probability distributions differ. The test compares the medians of two samples instead of their means. The differences between the median and each individual value for each sample is calculated. Values that come to zero are removed. Any remaining values are ranked from lowest to highest. Lastly, the ranks are summed. If the rank sum is different between the two samples it indicates statistical difference between samples.

%\begin{equation}
%\begin{aligned}
%    z &= \frac{W - 0.5}{\sqrt{\frac{n(n+1)(2n+1)}{6}}}, \\
%    W &= |\sum[sgn(x_2 - x_1)R]|,
%\end{aligned}
%\end{equation}
%with $x_i$ the data, $n$ the size of the samples and $R$ the rank.
%We set an arbitrary threshold for the result of this test. If the value of the Wilcoxon test exceeds the threshold, it means distributions are similar.

\subsubsection*{$\chi^2$ test}

The $\chi^2$ test is a statistical test used to determine whether or not there is a significant association between two distributions. There exist several variant of this test, we will use the most used one : the Pearson variant. It follows the formula : 
\begin{equation}
    \chi^2 = \sum_{i=1}^n = \frac{(O_i - E_i)^2}{E_i},
\end{equation}
with $O_i$ the number of observations of type i, $E_i$ the expected (theoretical) count of type i.
From there, we can compute the p-value. If this value is \textbf{less} than a specified threshold (for instance, $0.05$), we conclude a significant association between the two populations. Once again, the $\chi^2$ statistic is provided by the \textit{scipy} function, it represents the the discrepancy between the observed and expected frequencies. If this value is large, the two distributions are considered as different.

\subsection{Mutual Information}

The mutual information (MI) is a commonly used quantity in information theory that measures the mutual dependence between two random variables. It quantifies the amount of information obtained about one variable by observing the other variable. The MI is tightly bounded to the concept of \textit{entropy}, a notion that quantifies the expected amount of information held in a specific variable.\\
In this work, I prefer the use of MI over the score of correlation, since the former only measures linear dependency between variables \footnote{For instance, the correlation score between $\cos{x}$ and $\sin{x}$ is tiny, while their dependency to each other is obvious.}.\\
The mathematical formulation of MI in our case is :
\begin{equation}
    I(X;Y) = \sum_x \sum_y P_{(X,Y)} (x,y) \log \left( \frac{P_{(X,Y)(x,y)}}{P_X(x) P_Y(y)} \right),
\end{equation}
with $ P_{(X,Y)}$ the joint probability mass function of $X$ and $Y$, $P_X$ and $P_Y$ the marginal probability mass function of $X$ and $Y$ respectively.\\

\newpage

\section{Network development history}

The original code which inspired this work can be found at [\ref{github GAN}]. It is originally designed to be trained on the MNIST dataset and thus, produce sample of handwritten digits. In this kind of GANs, some parts are precisely dedicated to image processing, as convolutional layers or max pooling for example. These will be no use for this work, thus those are not implemented in our GAN. Moreover, we want to plot the evolution of loss function for both the generator and discriminator to monitor the learning process as well as plotting the learning rate over time. The latter plots will allow us to double check whether or not our adaptive learning rate techniques work as intended.\\

The whole process of fine-tuning hyperparameters and stabilizing the whole network represents a major task of this thesis. The estimated required time to achieve this is approximately ?? weeks.

\subsection{First results}

Right after the basic modifications to transition from image-related to data-related, here are the distributions generated by two different networks, with the same set of hyperparameters. The first one is characterized by a single hidden layer with 8 nodes, while the second one is made of 4 hidden layers with 256 nodes each.

\begin{figure}[H]
    \centering
    \begin{subfigure}{0.37\textwidth}
        \centering
        \includegraphics[scale = 0.45]{invMass_generated_simple_network.png}
        \label{complexe}
    \end{subfigure}
    \hspace{1.3cm}
    \begin{subfigure}{0.37\textwidth}
        \centering
        \includegraphics[scale = 0.45]{invMass_generated.png}
        \label{simple}
    \end{subfigure}
    \caption{Example of output for different networks. Left : simple network. Right : complex network.}
\end{figure}

\subsection{First convergences}

Once the architecture of the model set, we can try different set of hyperparameters. At first glance, these results look satisfying. However, due to several reasons, as a (very) reduced number of epochs for the training, this network is very unstable. As stated in a previous section, totally different output distributions can be generated for the exact same set of hyperparameters. Thus, the fine-tuning of hyperparamters is near impossible since no consistent benchmark exists.
This could be explained by the loss function being stuck in a local minimum, hence the different results obtained. Indeed, at this point, there is no weight initializer, no adaptive learning rate techniques, ... to address this issue.

\begin{figure}[H]
    \centering
    \begin{subfigure}{0.37\textwidth}
        \centering
        \includegraphics[scale = 0.37]{Development/5x16_0.00001_3000.png}
    \end{subfigure}
    \hspace{1.3cm}
    \begin{subfigure}{0.37\textwidth}
        \centering
        \includegraphics[scale = 0.47]{Development/5x16_0.001_3000.png}
    \end{subfigure}
    \caption{Example of output for a same network but with slighlty different learning rate}
\end{figure}

\subsection{Number of epochs and loss functions}

The first idea to get out of these local minima was to significantly expand ($\times 10-15$) the training of the network. Moreover, we plot the evolution of the loss function to gain a better understanding of the process.\\
Unfortunately, it modifies the shape of the output distribution, worsening the overall result. This is a terrible sign, increasing the training shouldn't drive the network away from the expected distribution, it should do the complete opposite. Sometimes even causing overfitting. In addition, the loss function can adopt unexpected behaviour, reinforcing our bad resentment.
Indeed, the decreases of both losses shown in Fig.(\ref{distrib + lr x2}c) and in Fig.(\ref{distrib + lr x2}d) are clearly different. It could be explained by the gradient being stuck in a local minimum or a saddle point in the case of Fig.(\ref{distrib + lr x2}d). It means that, unfortunately, increasing the number of epochs is not the solution to escape local minima and saddle points in our case, we need to find something else.
However, it doesn't mean the change has to be reverted. It might be a step in the right direction, however, there is still quite a run to go.

\begin{figure}[H]
    \centering
    \begin{subfigure}{0.37\textwidth}
        \centering
        \includegraphics[scale = 0.45]{Development/bad_distrib_convergence_v1.png}
    \end{subfigure}
    \hspace{1.3cm}
    \begin{subfigure}{0.37\textwidth}
        \centering
        \includegraphics[scale = 0.45]{Development/decent_distrib_stagnation_v2.png}
    \end{subfigure}
    \begin{subfigure}{0.37\textwidth}
        \centering
        \includegraphics[scale = 0.45]{Development/convergent_shaky_slope.png}
    \end{subfigure}
    \hspace{1.15cm}
    \begin{subfigure}{0.37\textwidth}
        \centering
        \includegraphics[scale = 0.45]{straight_slope_CE.png}
    \end{subfigure}
    \caption{Example of output for a same divergent network with the corresponding loss function (binary cross-entropy) evolution.}
    \label{distrib + lr x2}
\end{figure}

\subsection{Encouraging follow-up ?}
To address this problem of inconsistency, adaptive learning rate methods (such as cyclic LR, LR scheduler or reduce LR on plateau) were considered. Here are the generated distributions : Fig.(\ref{follow up}). It does not look obvious, but with these improvements the network is now stable. The fine-tuning is then much easier to perform.\\
However, the network remains far from the expected distribution.\\
From now on, the loss function always adopt a similar shape to the lower-left plot at Fig.(\ref{distrib + lr x2}).

\begin{figure}[H]
    \centering
    \begin{subfigure}{0.37\textwidth}
        \centering
        \includegraphics[scale = 0.45]{Development/5x32_CyclicLR_30k.png}
    \end{subfigure}
    \hspace{1.3cm}
    \begin{subfigure}{0.37\textwidth}
        \centering
        \includegraphics[scale = 0.45]{distrib_with_scheduler.png}
    \end{subfigure}
    \caption{Implementation of adaptive LR methods. Left : Cyclic (triangular). Right : Scheduler.}
    \label{follow up}
\end{figure}

\subsection{Encouraging follow-up !}

After the implementation of \textit{He} weight initializer, of batch normalization with a greater batch size coupled to another cyclic learning rate strategy (decreasing triangular), here is the output of the GAN.
Despite a clear improvement on the previous case, the generated distribution remains at a distance from the input, mainly because the generated simulation is shifted to the right in comparison to the inputs.

\begin{figure}[H]
    \centering
    \includegraphics[scale = 0.45]{Development/10_04_best_result_so_far.png}
    \caption{Output provided by an almost converging network}
    \label{fig:enter-label}
\end{figure}

\subsection{Final result for 1DGAN}

Once the shifting problem solved using more adapted rescaling techniques, we obtain the following result for the 1DGAN :

\begin{figure}[H]
    \centering
    \includegraphics[scale = 0.45]{Development/best_result_for_MuonPt.png}
    \caption{Best result obtained so far.}
    \label{fig:enter-label}
\end{figure}

%So far, it has been fairly easy to compare the generated with the input distributions using only histograms. However, these two samples being very similar, visual comparison is too weak to base our analysis on. From now on, the \textit{Kolmogorov-Smirnov test} [\ref{KS test}] and the \textit{Z-test} [\ref{Z test}] will also be used to determine the similarity between the two distributions we are interested in.

\subsection{2-dimensional GAN}

As one can see on Fig.(\ref{2dGAN}), the network converges in a satisfying manner to the input distributions. Although, the generated populations show difficulty to match peaks of the initial sample.

\begin{figure}[H]
    \centering
    \includegraphics[scale = 0.37]{Development/18_04_best_so_far.png}
    \caption{Output provided by a 2D GAN. The variables generated are the tranverse momentum of muons and the invariant mass of the system}
    \label{2dGAN}
\end{figure}

One-dimensional histograms are good tools to check how close to the initial distribution our generated sample stands. However, the main goal of our network is not only to replicate the inputs, but also to mimic the correlations between the variables set as input.\\
To visualize these dependencies, we use two-dimensional histograms as shown in Fig.(\ref{hist2D}). On the first plot, the correlation between the variable "invMass" and "MuonsPt" are shown for the generated data, while for the following plot, it is the correlation between the same variables but for input data. According to these two previous plots, I can assume that the network is doing a satisfying job in reproducing correlations.\\


\begin{figure}[H]
    \centering
    \includegraphics[scale = 0.42]{MuonPt_corr.png}
    \caption{Correlations between the two variables. Two zones stand out, the biggest corresponds to Drell-Yan events where a $Z$ is the mediator boson, while the other stands for a $\gamma$ as mediator boson.}
    \label{hist2D}
\end{figure}

\subsection{3-dimensional GAN}

\begin{figure}[H]
    \centering
    \includegraphics[scale = 0.45]{Development/19_04_3D_GAN.png}
    \caption{Output provided by a 3D GAN. The variables generated are the tranverse momentum of muons, the invariant mass of the system and the missing transverse momentum.}
    \label{fig:enter-label}
\end{figure}

\begin{figure}[H]
    \centering
    \includegraphics[scale = 0.4]{Development/3D_Hist2D_vanilla.png}
    \caption{Correlations between the two variables generated by the network.}
    \label{fig:enter-label}
\end{figure}

I decide to use the mutual information of two variables to compute the dependency between them. The result obtained is : 

\begin{table}[h]
    \centering
    \begin{tabular}{|l|c|c|}
        \hline
        \textbf{Mutual Information score} & \textbf{Outputs} & \textbf{Inputs} \\
        \hline
        MI(MuonPt, invMass) & 9.3875 & 8.6731 \\
        MI(invMass, MET\_pt) & 9.3880 & 8.6733 \\
        MI(MuonPt, MET\_pt) & 9.3908 & 8.6731 \\
        \hline
    \end{tabular}
    \caption{Mutual Information Values for a 3-D GAN.}
    \label{tab:mi_values}
\end{table}

\newpage

\subsection{Conditional GAN}

The next step is to transition to a conditional GAN, which is the goal of this work. The idea is simply to provide a new piece of information to the generator, called a "label". This label refers to the type of data that the network will be generating. We select the presence of b-tagged jets as the label, thus creating two classes : without b-jets and with at least one b-jet. It is also the variable blinded in the training sample.\\

First, let's check how is the network performing on each variables.\\

\begin{figure}[H]
    \centering
    \includegraphics[scale = 0.45]{Development/19_04_3D_GAN.png}
    \caption{PLACEHOLDER}
    \label{fig:enter-label}
\end{figure}

In addition to these kind of visual tool, we also use the \textit{Kolmogorov-Smirnov test} (KS) and the $\chi^2$ \textit{test} to assess how similar/different the two distributions evaluated are. For the KS, we obtain : 
\begin{table}[H]
    \centering
    \begin{tabular}{|l|c|c|}
        \hline
        \textbf{KS} & \textbf{p-values} & \textbf{statistics}  \\
        \hline
        MuonPt & 0.007 & 0.433\\
        invMass & 0.007 & 0.433\\
        MET\_pt & 0.239 & 0.267\\
        \hline
    \end{tabular}
    \caption{KS test between inputs and outputs for each observable.}
\end{table}
If the threshold is set at $5\%$ signifiance, only the MET\_pt generated distribution would be considered similar to the training sample corresponding observable.

and for $\chi^2$ :
\begin{table}[H]
    \centering
    \begin{tabular}{|l|c|c|}
        \hline
        \textbf{$\chi^2$} & \textbf{p-values} & \textbf{statistics}  \\
        \hline
        MuonPt & 0 & 668.134\\
        invMass & 0.00002 & 71.200\\
        MET\_pt & 0.00003 & 70.594\\
        \hline
    \end{tabular}
     \caption{$\chi^2$ test between inputs and outputs for each observable.}
\end{table}

As seen in Fig.(\ref{all corr}), we can visualize the dependencies between the three selected variables for the generated data and the training sample. Although the zones for outputs are larger, the general behaviour of the dependencies remains similar.

\begin{figure}[H]
    \centering
    \includegraphics[scale = 0.42]{invMass_corr.png}
\end{figure}

\vspace{-\baselineskip} % Reduce space between figures

\begin{figure}[H]
    \centering
    \includegraphics[scale = 0.42]{MuonPt_corr.png}
\end{figure}

\vspace{-\baselineskip} % Reduce space between figures

\begin{figure}[H]
    \centering
    \includegraphics[scale = 0.42]{MET_pt_corr.png}
    \caption{Comparison of the input and output correlations between the different variables used.}
    \label{all corr}
\end{figure}
To get numerical values of the dependencies between observables, we can also compute the mutual information values for each pair of variables, the obtained values are :

\begin{table}[H]
    \centering
    \begin{tabular}{|l|c|c|}
        \hline
        \textbf{Mutual Information score} & \textbf{Outputs} & \textbf{Inputs} \\
        \hline
        MI(MuonPt, invMass) & 10.3075 & 8.6731 \\
        MI(invMass, MET\_pt) & 10.3080 & 8.6733 \\
        MI(MuonPt, MET\_pt) & 10.2908 & 8.6731 \\
        \hline
    \end{tabular}
    \caption{Mutual Information Values for a 3D cGAN.}
    \label{tab:mi_values}
\end{table}
These numbers might be hard to interpret, since the mutual information score ranges from 0 to $+\infty$. However, the exact values are not the important detail to remember, the similarity between the variables of the input and the output is. Indeed, in both cases, the MI score orbits around the same value, meaning that the correlations between observables is close to our goal, despite being slightly larger than expected.

\subsection{CMS data}

The final step is to apply the network to actual data, or not on a sample generated by MC simulation. The nTuple used contains data collected in 2022 by CMS, for approximately 13 000 events and with all leptons being muons.\\
Here are the variables specified in the nTuple :

\begin{figure}[H]
    \centering
    \begin{minipage}{0.45\textwidth}
        \centering
        \includegraphics[width=\textwidth]{Development/Figure 2024-05-18 173119 (2).png} % Replace with your image path
        
        \label{fig:figure1}
    \end{minipage}
    \hfill
    \begin{minipage}{0.45\textwidth}
        \centering
        \includegraphics[width=\textwidth]{Development/Figure 2024-05-18 173119 (0).png} % Replace with your image path
        
        \label{fig:figure2}
    \end{minipage}

    \vspace{1em} % Space between rows

    \begin{minipage}{0.45\textwidth}
        \centering
        \includegraphics[width=\textwidth]{Development/Figure 2024-05-18 173119 (1).png} % Replace with your image path
        
        \label{fig:figure3}
    \end{minipage}
    \hfill
    \begin{minipage}{0.45\textwidth}
        \centering
        \includegraphics[width=\textwidth]{Development/Figure 2024-05-18 173119 (4).png} % Replace with your image path
        
        \label{fig:figure4}
    \end{minipage}
    \caption{Generated variables in the CMS nTuple.}
\end{figure}


\newpage

\section{Numerical methods}

As stated many times in this work, the background modeling is a crucial component to the success of an experiment in high energy physics. To achieve this, several strategies are available : data-driven methods, Monte Carlo simulations, parametric and non-parametric methods among others. For basic cases, data-driven background estimation methods are sufficient to estimate the expected number of background events. Unfortunately, these basic cases only represents a minority. In most cases, background modeling relies on direct simulation based on Monte Carlo (MC) event generators or parametric method.\\
However, for even more intricate situations, these two techniques fall short. The problem is, despite a sufficient background modelling concerning some of its components, it is not enough to accurately predict the characteristics of the total background. That's the problem tackled in this thesis.\\
Indeed, in this work we address the background modelling of the Drell-Yan process, for di-Higgs physics. The other components of the total background, i.e $t \Bar{t}$ production and $W+jets$ events, can be fairly easily modelled. However, this isn't the case for DY due to its varying final state signature. To achieve our goal, we use a \textit{non-parametric data-driven background modelling} method, via a cGAN. Let's briefly breakdown what this expression actually means.

\subsection{Parametric and non-parametric methods}

\subsubsection*{Parametric}
Parametric methods are statistical techniques relying on specific assumptions about the underlying distribution of the population being studied. These methods typically assume that the data follows a known probability distribution, such as the normal distribution, and estimate the parameters of this distribution using the available data. Parametric methods are those methods for which we a priori know that the population follows a Gaussian distribution, or if not then we can easily approximate it using such a distribution. In addition to the Gaussian assumptions, these techniques also assume the independence between observations aswell as a homogeneous variance over the set of events.
For normal distributions, the parameters are : the mean ($\mu$) and the standard deviation ($\sigma$).\\

The efficency of this category of methods heavily relies on whether or not the assumptions are met. If that's the case, these techniques are very powerful (i.e. able to detect a real effect when it exists), even working on a reduce set of events. However, it makes those a very rigid option, which may not capture complex relationships between variables.

\subsubsection*{Non-parametric}

Conversely to the previous category of methods, non-parametric ones do not rely on specific assumptions of parameters. In fact, they don't depend at all on the population studied. Hence, there are no parameters or distributions needed. However, some assumptions about the data are still required as the independence of observations or the homogeneity of measurements.
Conversely to parametric methods, non-parametric ones are widely applicable due to their independence to the studied population, their easy implementation and their robustness to outliers.
However, when the assumptions of parametric methods are met, these ones remain more powerful and require smaller sample to achieve the same level of power.\\

Some examples of non-parametric methods used for LHC data analysis are Kernel Density Estimation (KDE) used to estimate the probability density function of a random variable based on a sample of data. It can be used to visualize the distribution of a specific observable without the assumption of a parametric form for the distribution. Moreover, Random Forest algorithms (an advanced version of decision trees) are also non-parametric approaches. These machine learning algorithms are used to perform classification and regression tasks based on the characteristics of the events/particles probed, without, once again, assuming a specific parametric form for the underlying data distribution.

In this thesis, complex relationships are expected among the variables simulated by our GAN. Hence, we choose to work with non-parametric methods due to their flexibility and low-computational cost.

\subsection{Data-driven methods}

Data-driven methods are a class of methods that primarily rely on current data collected during the system's/process' lifetime in order to establish relationships between input, internal and output variables. Their aim is to efficiently process and analyze large datasets. Hence their usefulness for the generalization of our cGAN to samples of considerable size.\\
The term data-driven modeling refers to the use of current data merged with advanced computational techniques, as machine learning, to create models revealing underlying trends and patterns between variables of a same dataset. Moreover, data-driven models can be built with or without detailed knowledge of the underlying processes governing the system behavior, which makes them particularly useful when such knowledge is not in our possession. Hence, data-driven background estimates are a must in situations where you cannot get a reliable estimate from simulation.\\

Such methods are extensively used in the scope of data analysis at the LHC. For jet mass reconstruction, some very important variables are determined thanks to data-driven approach. Indeed, both ATLAS and CMS developped tagging algorithms for jets, that includes an array of validation and calibration techniques processed in a data-driven manner.

\subsubsection{ABCD method}

The ABCD method [\ref{abcd}] is a common use to get data-driven background estimation, as seen in [\ref{Agni bbww}] one of the main paper this work is based on. The idea behind this method is illustrated in Fig.(\ref{fig:abcd}).
The phase space is divided into four different regions, each defined by variables uncorrelated to each other. The region D is the \textit{signal region}, in other words, the phase space region defined by the triggers and selections used for the signal we are interested in. While the other regions (A, B, C) are the \textit{control regions}, these are obtained by modifying some of the cuts used for the signal selection, in order to obtain similar regions to the signal one, with the important difference that control regions do not contain any signal \footnote{In an ideal case, there is no signal at all. However, some signal might be present in these control regions, but the ratio signal-over-background ratio will remain tiny.}.  Control regions are usually defined over a specific background process, with enough events to insure sufficient statistics. The shape of the background process can then be estimated as a function of one or several variables.
These regions can also be referred as \textit{sidebands} in cases where the signal appears as a resonance peak. Signal region is then a specific window and the control regions are on both sides on this windows, hence the name sidebands.

Although the control regions are defined to be as similar as possible to the signal region, some differences in the selection efficiency for the background process may happen between these two regions. Thus, the control region is corrected by deriving additional events weights called \textit{transfer factors}. To determine these factors, we use the two remainings regions : A and B. We assume that the ratio between A and B is defined by the same cuts than the ratio between C and D. Transfer factors are then determined by the change of background from A to B, and they are then applied to C.

\begin{figure}[h]
    \centering
    \includegraphics[scale=0.5]{ABCD_method.png}
    \caption{Representation of the ABCD method}
    \label{fig:abcd}
\end{figure}

\subsection{Shortcomings of Monte Carlo simulations}

The Monte Carlo (MC) simulations are a broad class of computational algorithms.\footnote{The intrinsic operation of MC simulations won't be addressed here, since it is of no use in this work. For more information, please see [\ref{Monte Carlo}].} These simulations are very widespread in many fields, such as high energy physics.
Despite their numerous advantages, MC techniques hold several heavy limitations, common to all their applications. These include : high computational cost, especially with complex model accounting many variables; heavy reliance on quality input data and on the different assumptions made; aswell as a difficult result interpretation, especially for cases without a strong statistical background. Moreover, when it comes to high energy physics, there also problems coming from imperfect modelling of the detectors and limitations due to fixed-order calculations, meaning that only a finite number of terms in the perturbative expansion are considered.\\
For all those reasons, we cannot entirely rely on MC simulations. Hence the need of the new approach discussed in this work.

\subsection{b-tagging algorithms}

The identification (tagging) of the jets coming from the hadronization of heavy flavor quarks, such as bottom quarks, is made possible by the use of data-driven approaches and by distinctive properties of the heavy hadrons. For instance, B-hadrons have a relatively large lifetime ($\mathcal{O}(1.5ps)$) leading them to travel measurable flight length path of a few millimeters before their decay into lighter hadrons. Thus, creating a secondary vertex clearly distinct from the main one.

\begin{figure}[H]
    \centering
    \includegraphics[scale = 0.19]{B-tagging_diagram.png}
    \caption{Secondary vertex from a b-jet}
    \label{fig:enter-label}
\end{figure}

Moreover, B-hadrons are massive which leads to decay products with larger tranverse momentum (relative to the jet axis) in comparison to jets produced by lighter partons.

B-tagging algorithms have become a crucial components in current data analysis. For instance, for the search of the Higgs boson in the $t\Bar{t}H$ channel, these algorithms were used to reduce or even eliminate large backgrounds like $t\Bar{t}j\Bar{j}$ or $W+jets$. Indeed, the $t\Bar{t}j\Bar{j}$ background was reduced by two orders of magnitude only using b-jets identification.\\
When it comes to current experiments, such algorithms remain a crucial component for a successful analysis. Indeed, in the channel probed in this work ($\Bar{b}bW^+W^-$), top quarks are decaying into $b$'s, hence the need of b-tagging algorithms. Moreover, the additional label information used by the cGAN directly depends on b-tagging information since it is a solid criterion to separate the signal region from the background one. Indeed, DY events are not expected to produce two b-jets.


\subsection{Morphing}

In high energy physics, "morphing" refers to a technique used to interpolate between different simulated events or physical models seamlessly. It allows the exploration of the behaviour of a physical system over a continuous parameter space without the need of plenty of discrete, independent simulations at each point in the parameter space. Morphing techniques typically involve constructing a parameterized mapping between the original and target parameter spaces, often using mathematical functions or interpolation methods. This mapping allows for the generation of simulated events corresponding to intermediate parameter values, providing a more comprehensive understanding of the physics being studied.\\

In the LHC, the use of morphing techniques for systematic uncertainties is a very common thing. These methods are used to assess the impact of systematic uncertainties on measurements of several parameters as mass, cross-section, ... For example, for uncertainties of the simulation of a physical process, morphing techniques can smoothly interpolate to different simulation in order to estimate the effect of these uncertainties on the final result. [\ref{morphing}]

\subsection{Application of the cGAN-based approach}

In the paper which this thesis is based on, an ATLAS search [\ref{atlas z}] for Higgs boson decaying to a $Z$ boson and a light hadronically decaying resonance $a$ is taken as case study. In the ATLAS original paper, several of the above techniques are used to perform the analysis. Indeed, a variant of the ABCD method and a MC-based method are used to account for the correlation between several variables.  This search is facing two main obstacles : first, for a $\mathcal{O}(100fb^{-1})$ dataset, it is impossible to generate a simulated event sample with a comparable statistical power. Second, the decaying resonance $a$ is identified with a multi-variate methods, requiring a detailed modelling of a several correlations between variables related to kinematics and jets. The sensitivity of the initial search is thus limited by systematic and statistical uncertainties. 
%The sensitivity of the initial search is thus limited by the background systematic uncertainties, mainly coming from the insufficient size of the simulated data samples used. 

Both of these obstacles come from the insufficient size of simulated samples implied by the techniques used originally. However, the cGAN-based approach is able to overcome these. Indeed, with the possibility of generating larger samples, the statistical uncertainties could be suppressed. Thus, allowing a performance improvement, as proved in the case study mentionned. There will still be some remaining uncertainties coming from the training of the cGAN. To mitigate these, we plan to run several networks (5, typically) and combine their results.
Moreover, the cGAN is able to target a region (e.g. signal region, background region,...) in which the data will be generated, which is impossible with the current methods. This approach could also be directly based on real data and not MC simulations. Indeed, we could use the official CMS data as a training sample for the network.

This is exactly what this work aims to do, but for a different purpose. As mentionned in [\ref{Agni bbww}], for the di-Higgs physics $b \Bar{b}W^+W^-$ final state, several backgrounds are considered such as : $t \Bar{t}+jets$, single-top production, $WW$ processes as well as Drell-Yan among others. Despite not being generated with the same softwares (\textit{MadGraph5\_aMC@NLO} for DY, \textit{POWHEG} [\ref{POWHEG}] for the others mentionned), these backgrounds components are still coming from the same simulation strategy : Monte Carlo. In addition, a very similar approach to the ABCD method is also used to estimate DY events. We want to assess whether or not the cGAN approach is viable alternative to the "MC + ABCD + morphing" combination for the $b \Bar{b}W^+W^-$ case.\\

\newpage

\section{GANs and other generative modelling methods}

%\textbf{trade-off accuracy/recall/jsp quoi}

Normalizing flows (NFs) [\ref{NF}] are a class of models used in machine learning for generative modelling. The main idea behind normalizing flows is to transform a simple probability distribution, such as a standard Gaussian distribution, into a more complex distribution that closely matches the true distribution.\\ 
Normalizing flows consist of a series of invertible transformations applied to a simple base distribution. These transformations are designed to gradually deform the base distribution in order to converge to the expected result. Each transformation in the flow must be invertible, meaning that you can easily compute both the forward and inverse transformations.  By chaining together several of these transformations, normalizing flows can model complex data distributions with intricate patterns and dependencies. The final distribution obtained after applying all transformations is a complex, non-Gaussian distribution which should, as closely as possible, matches the data distribution.\\

\begin{figure}[H]
    \centering
    \includegraphics[scale = 0.7]{normalizing-flow.png}
    \caption{Operation of a normalizing flow}
    \label{Operation of a normalizing flow}
\end{figure}

In comparison to other generative models, such as GANs and variational autoencoders (VAEs) [\ref{VAE}] , the training process of NFs is much more stable, it is not required to thoroughly fine-tune hyperparameters. Moreover, flow-based algorithms tend to converge faster than other models. However, very high-dimensional latent vectors are necessary, which is usually hard to interpret. Let's go through a short state of the art about this strategy.

%Moreover, on image-related datasets, the samples generated with flow-based algorithms are not as good when compared to GANs and VAEs. Despite these shortcomings, it is important to remember that the field of NFs is still in relative infancy, and is expected to progress quickly in the coming years.\\

Recently, several searches have monitored the efficiency of these generative models on similar tasks in order to compare them efficiently.\\
First, in 2021, an empirical comparison between GANs and NFs was made [\ref{GAN vs NF}], on different datasets but, with the common point of being low-dimensional data. Both a standard GAN and a WCGAN [\ref{WCGAN}] were used to draw this comparison. Surprisingly, the NF outperforms both type of GANs on several metrics, as kernel density estimation or the very metric on which the WCGAN is based on : the Wasserstein-1 distance. However, this analysis only holds to low-dimensional datasets. There is no certainty that this conclusion could be translated to high-dimensional ones.\\
Then, in 2022, a comparison between several generative modelling strategies, such as GANs, NFs and VAEs,  was performed. [\ref{GAN NF VAE}] The GAN and its variations are presented as the best performing algorithms in terms of training and test speed, parameter efficiency, sample quality, sample diversity, and ability to scale to high resolution data. However, the gap between GANs and the latest version of other approaches was shrinking at the time the paper was published.\\ 
Eventually, a search about speech enhancement published in late 2023 [\ref{speech}] takes an interesting approach. Indeed, a new variation of GAN is derived : the SEFGAN. The main idea behind this variant is the use of a NF as generator for the GAN. This article shows that this hybrid approach clearly outperforms a pure NF or pure GAN strategy according to computational metrics and listening experiments done by human listeners.

\newpage

\section{Conclusion}

\textbf{null hypothesis or alt. ?? + normalized or not distrib ? + talk about cross validation, k-fold and checking overfit + non-perturbative QCD}

\begin{itemize}
    \item Satisfying results, even if difficulty to match peaks
    \item results on CMS data ???
    \item Decent computational requirements
    \item results not perfect but I strongly believe that it could be improved with longer training and with optimal hyperparameters, not done here because it requires heavy computational power
    \item loss of accuracy while generating more observables, might be a hard limit for this approach. But, generating 35 variables shouldn't be that useful, simulating the most basic ones like MuonPt, Eta, (phi?), invMass should be enough for plenty of different tasks
    \item way of improvement : SEFGAN (GAN + NF) or GAN + transformer instead of the generator (DNN initially)
    \item sum up ? "I presented the challenge of finding an evidence of the 2HDM, quickly detailed the targetted channel, its advantages and disadvantages over other existing decay channels, the background to overcome in order to detect the signal, how the current techniques struggle to filter that bkg out and finally, introduced the new cGAN approached and its result on (both) MC simulation (and CMS data)"
\end{itemize}

The goal of this work was to briefly introduce the 2HDM and to detail the different mechanisms of Higgs bosons pair production and decay. what are the obstacles encountered to validate this theory with an experimental evidence.\\

The results obtained with this new cGAN approach seem promising, I have been able to overcome the main weakness of the GAN-like algorithms by using different tools, in order to make the network converge to a solution. The generated samples are similar to the target distributions, despite showcasing some weaknesses, especially for sharp peaks. I strongly believe that a more effective training and a set of more adapted hyperparameters could easily solve this issue. Indeed, the training of the network was purposefully kept short for time considerations. When it comes to hyperparameters, a plausible improvement would be to create an hypergrid containing all parameters, to evaluate these selections in order to choose the most suited set of hyperparameters to the task. However, it would require heavy computational power, which I had no access to.\\

In my opinion, the most limitating factor for the cGAN would be the number of variables generated. Indeed, each time the dimension of the network has been increased, it resulted in an accuracy loss. The effects were not significant, but these could quickly become overwhelming for a large number of variables. In this work, I went until three variables, knowing that training effectively a 4(or more)D cGAN would be out of my reach. In the paper this thesis is based on, the research team was able to reach five variables with convincing results. 

One way of improvement would be to replace the DNN used as the cGAN generator by a generative machine learning algorithm, such as a normalizing flow or a transformer. The former has already been trialed in other fields than particle physics and shows encouraging results. About the later, sole transformers are already providing promising results in the field of high energy physics.

\newpage

\appendix

\section{Skills acquired during the project}

During this thesis, I had to use an array of new concepts and tools. Here is a summary :

\paragraph{Software}
\begin{itemize}
    \item MadGraph and one of its model, the 2HDM, to generate the initial training sample
    \item ROOT (on Ubuntu), to open the different .root files used and to check their content
    \item Delphes, to simulate a (simplified) detector response, still for the initial sample
\end{itemize}

\paragraph{Packages}
\begin{itemize}
    \item Keras, for machine learning
    \item TensorFlow, for machine learning
    \item uproot, for manipulating .root files
    \item scipy, for tools as mutual information and KS
    \item awkward arrays, type of arrays used in the .root files
\end{itemize}

\paragraph{Machine learning}
\begin{itemize}
    \item GAN, cGAN and "blinded" cGAN
    \item brief introduction to the concept of normalizing flows
    \item plenty of different tools to ease the convergence of a network : dynamic learning rate techniques, \textit{He} initialization, gradient clipping, ...
\end{itemize}

\paragraph{Soft skills}
\begin{itemize}
    \item better english skills, both active and passive
\end{itemize}

\newpage

\section{Bibliography}
\begin{enumerate}
    %Global sources
    \item Uliège thesis \url{https://orbi.uliege.be/bitstream/2268/68445/1/MmemoireFinal.pdf} \label{uliege}
    \item Florian's thesis
    \item Main article
    \item Standard model and beyond course
    \item Machine learning course
    \item Fundamental interactions and elementary particles course
    \item Particle accelerator $\&$ neutrino course
    %H -> bbWW Agni's paper
    \item Search for HH production in the bbW+W- decay mode in proton-proton collisions at s = 13 TeV The CMS Collaboration \label{Agni bbww}
    %Higgs boson
    \item Amsler, C., et al. « Review of Particle Physics ». Physics Letters B, vol. 667, no 1‑5, septembre 2008, p. 1‑6. DOI.org (Crossref), \url{https://doi.org/10.1016/j.physletb.2008.07.018}. \label{vev}
    \item InspireHEP. (n.d.). File details - 6dec6d5b7461dfa2693c895eafc4f711. Retrieved April 14, 2024, from \url{https://inspirehep.net/files/6dec6d5b7461dfa2693c895eafc4f711} \label{higgs decay}
    %Neutrinos
    \item Dvorkin, Cora, et al. Neutrino Mass from Cosmology: Probing Physics Beyond the Standard Model. arXiv:1903.03689, arXiv, 8 mars 2019. arXiv.org, \url{https://doi.org/10.48550/arXiv.1903.03689}. \label{neutrino mass}
    %antimatter
    \item Sakharov, Andrei D. « Violation of CP in Variance, C Asymmetry, and Baryon Asymmetry of the Universe ». Physics-Uspekhi, vol. 34, no 5, mai 1991, p. 392‑93, \url{https://ufn.ru/en/articles/1991/5/h/}. \label{Sakharov}
    %Dark Matter
    \item Biermann, Peter L., et Faustin Munyaneza. « The Nature of Dark Matter ». AIP Conference Proceedings, vol. 972, 2008, p. 365‑73. arXiv.org, \url{https://doi.org/10.1063/1.2870344}. \label{cold DM}
    \item Frost, James. Dark Matter Searches at the LHC. 2022. CERN Document Server, \url{https://cds.cern.ch/record/2843045} \label{DM1}
    \item Einasto, Jaan. Dark Matter. arXiv:0901.0632, arXiv, 19 octobre 2010. arXiv.org, \url{http://arxiv.org/abs/0901.0632}. \label{DM2}
    \item Garrett, Katherine (2010). "Dark matter: A primer". Advances in Astronomy. 2011 (968283): 1–22. arXiv:1006.2483 \label{wimps}
    %Hierarchy problem
    \item Smith, E. (2019). The Hierarchy Problem. The University of Chicago, QFT III Final Paper,\url{https://homes.psd.uchicago.edu/~sethi/Teaching/P445-S2019/Emily_Smith_QFT_III_Final_Paper.pdf} \label{Hierarchy}
    \item  « CMS mesure la masse du Higgs avec une précision inédite ». CERN, 3 avril 2024, \url{https://home.cern/fr/news/news/physics/cms-measures-higgs-bosons-mass-unprecedented-precision} \label{higgs mass}
    \item Tanabashi, M., et al. « Review of Particle Physics* ». Physical Review D, vol. 98, août 2018, p. 030001. NASA ADS, \url{https://doi.org/10.1103/PhysRevD.98.030001}. \label{W mass}
    %types of 2HDM
    \item Branco, G. C., et al. « Theory and phenomenology of two-Higgs-doublet models ». Physics Reports, vol. 516, no 1‑2, juillet 2012, p. 1‑102. arXiv.org, \url{https://doi.org/10.1016/j.physrep.2012.02.002}. \label{types of 2HDM}
    \item CMS Collaboration. « Search for flavor-changing neutral current interactions of the top quark and Higgs boson in final states with two photons in proton-proton collisions at $\sqrt{s} =$ 13 TeV ». Physical Review Letters, vol. 129, no 3, juillet 2022, p. 032001. arXiv.org, \url{https://doi.org/10.1103/PhysRevLett.129.032001}. \label{FCNC}
    \item Branco, G. C., et al. « Theory and phenomenology of two-Higgs-doublet models ». Physics Reports, vol. 516, no 1‑2, juillet 2012, p. 1‑102. arXiv.org, \url{https://arxiv.org/pdf/1106.0034.pdf} \label{type 3 bis}
    \item Arhrib, A., et al. « Two-Higgs-Doublet type-II and -III models and $t\to c h$ at the LHC ». The European Physical Journal C, vol. 76, no 6, juin 2016, p. 328. arXiv.org, \url{https://doi.org/10.1140/epjc/s10052-016-4167-9}. \label{type 3}
    %SUSY
    \item Wikipedia contributors. (n.d.). Supersymmetry. In Wikipedia. Retrieved April 14, 2024, from \url{https://en.wikipedia.org/wiki/Supersymmetry}.
    \item Csaki, Csaba. « The Minimal Supersymmetric Standard Model (MSSM) ». Modern Physics Letters A, vol. 11, no 08, mars 1996, p. 599‑613. arXiv.org, \url{https://doi.org/10.1142/S021773239600062X}. \label{mssm}
    %Di-higgs
    \item Spira, Michael. « Effective Multi-Higgs Couplings to Gluons ». Journal of High Energy Physics, vol. 2016, no 10, octobre 2016, p. 26. arXiv.org, \url{https://doi.org/10.1007/JHEP10(2016)026}. \label{33.47}
    %NN_intro
    \item Wikipedia contributors. (n.d.). Neural network (machine learning). In Wikipedia. Retrieved April 14, 2024, from \url{https://en.wikipedia.org/wiki/Neural_network_(machine_learning)}. \label{NN_wiki}
    %concept of neurons
    \item Baeldung. (n.d.). Neural Networks and Neurons in Java. Retrieved April 14, 2024, from \url{https://www.baeldung.com/cs/neural-networks-neurons}. \label{concept of neurons}
    %Backpropagation /
    %Exp/van grad 
    \item Ioffe, Sergey, et Christian Szegedy. Batch Normalization: Accelerating Deep Network Training by Reducing Internal Covariate Shift. arXiv:1502.03167, arXiv, 2 mars 2015. arXiv.org, \url{http://arxiv.org/abs/1502.03167}. \label{batch}
    %Overfitting /
    %LR + SGD
    \item Bisong, E. (n.d.). Optimization for Machine Learning: Gradient Descent. In Optimization for Machine Learning (pp. 281-303). Retrieved April 14, 2024, from \url{https://link.springer.com/chapter/10.1007/978-1-4842-4470-8_16}. \label{learning rate + sgd}
    %GAN
    \item B. N. S. Reenu. (n.d.). Python for Microscopists. GitHub. Retrieved April 14, 2024, from \url{https://github.com/bnsreenu/python_for_microscopists}.\label{github GAN}
    \item Eckerli, Florian, et Joerg Osterrieder. Generative Adversarial Networks in finance: an overview. arXiv:2106.06364, arXiv, 6 juillet 2021. arXiv.org, \url{https://arxiv.org/pdf/2106.06364.pdf} \label{Finance}
    \item Goodfellow, Ian J., et al. Generative Adversarial Networks. arXiv:1406.2661, arXiv, 10 juin 2014. arXiv.org, \url{https://arxiv.org/abs/1406.2661} \label{GAN original}
    \item  "THE MNIST DATABASE of handwritten digits". Yann LeCun, Courant Institute, NYU Corinna Cortes, Google Labs, New York Christopher J.C. Burges, Microsoft Research, Redmond. \label{MNIST}
    \item CIFAR-10 and CIFAR-100 datasets. \url{https://www.cs.toronto.edu/~kriz/cifar.html}. Consulté le 14 avril 2024.\label{CIFAR10}
    %Leaky ReLU 
    \item Xu, J., Li, Z., Du, B., Zhang, M., & Liu, J. (n.d.). Reluplex made more practical: Leaky ReLU. School of Software Engineering, Tongji University, Shanghai, China; Department of Computer Science, University of Warwick, Coventry, United Kingdom; School of Software Engineering, East China Normal University, Shanghai, China, \url{https://ieeexplore.ieee.org/stamp/stamp.jsp?tp=&arnumber=9219587} \label{leaky relu}
    \item Hendrycks, Dan, et Kevin Gimpel. Gaussian Error Linear Units (GELUs). arXiv:1606.08415, arXiv, 5 juin 2023. arXiv.org, \url{http://arxiv.org/abs/1606.08415}. \label{GELU}
    \item Clevert, Djork-Arné, et al. Fast and Accurate Deep Network Learning by Exponential Linear Units (ELUs). arXiv:1511.07289, arXiv, 22 février 2016. arXiv.org, \url{http://arxiv.org/abs/1511.07289}. \label{ELU}
    \item Klambauer, Günter, et al. Self-Normalizing Neural Networks. arXiv:1706.02515, arXiv, 7 septembre 2017. arXiv.org, \url{http://arxiv.org/abs/1706.02515}. \label{SELU}
    %Batch size
    \item Epochs, Batch Size, Iterations - How They Are Important. \url{https://www.sabrepc.com/blog/Deep-Learning-and-AI/Epochs-Batch-Size-Iterations}. Consulté le 14 avril 2024.
    %binary cross-entropy
    \item Ruby, U., & Yendapalli, V. (2020). Binary cross entropy with deep learning technique for image classification. International Journal of Advanced Trends in Computer Science and Engineering, 9(4). DOI: \url{10.30534/ijatcse/2020/175942020}. \label{binary cross entropy}
    \item Saxena, Shipra. « Binary Cross Entropy/Log Loss for Binary Classification ». Analytics Vidhya, 3 mars 2021,\\
    \url{https://www.analyticsvidhya.com/blog/2021/03/binary-cross-entropy-log-loss-for-binary-classification/}.
    %SGD cfr. SGD
    \item Amari, Shun-ichi. « Backpropagation and stochastic gradient descent method ». Neurocomputing, vol. 5, no 4, juin 1993, p. 185‑96. ScienceDirect, \url{https://doi.org/10.1016/0925-2312(93)90006-O}. \label{SGD v9}
    \item Kingma, Diederik P., et Jimmy Ba. Adam: A Method for Stochastic Optimization. arXiv:1412.6980, arXiv, 29 janvier 2017. arXiv.org, \url{https://arxiv.org/abs/1412.6980} \label{Adam}
    %Nesterov
    \item Xie, Xingyu, et al. Adan: Adaptive Nesterov Momentum Algorithm for Faster Optimizing Deep Models. arXiv:2208.06677, arXiv, 27 février 2023. arXiv.org, \url{https://arxiv.org/pdf/2208.06677.pdf} \label{Nesterov}
    %L2 regu /
    %Dropout layer /
    %He init
    \item He, Kaiming, et al. Delving Deep into Rectifiers: Surpassing Human-Level Performance on ImageNet Classification. arXiv:1502.01852, arXiv, 6 février 2015. arXiv.org, \url{https://arxiv.org/abs/1502.01852} \label{He init}
    %Clipping
    \item Pascanu, Razvan, et al. On the difficulty of training Recurrent Neural Networks. arXiv:1211.5063, arXiv, 15 février 2013. arXiv.org, \url{https://arxiv.org/abs/1211.5063} \label{clipping}
    %Preprocessing
    \item Ioffe, Sergey, et Christian Szegedy. Batch Normalization: Accelerating Deep Network Training by Reducing Internal Covariate Shift. arXiv:1502.03167, arXiv, 2 mars 2015. arXiv.org, \url{https://arxiv.org/pdf/1502.03167.pdf} \label{batch norm}
    %CLR
    \item Dauphin, Yann, et al. Identifying and attacking the saddle point problem in high-dimensional non-convex optimization. arXiv:1406.2572, arXiv, 10 juin 2014. arXiv.org, \url{https://arxiv.org/pdf/1406.2572.pdf}. \label{saddle}
    %cGan
    \item Mirza, Mehdi, et Simon Osindero. Conditional Generative Adversarial Nets. arXiv:1411.1784, arXiv, 6 novembre 2014. arXiv.org, \url{https://arxiv.org/abs/1411.1784}. \label{cGAN}
    %Sample
    \item MadGraph Development Team. (n.d.). Particle Content. Retrieved April 14, 2024, from \url{http://madgraph.phys.ucl.ac.be/particles.html}.\label{MadGraph}
    %Challenges
    \item Wang, Zhengwei, et al. Generative Adversarial Networks in Computer Vision: A Survey and Taxonomy. arXiv:1906.01529, arXiv, 29 décembre 2020. arXiv.org, \url{https://arxiv.org/pdf/1906.01529.pdf}. \label{hard to train}
    \item Salimans, Tim, et al. Improved Techniques for Training GANs. arXiv:1606.03498, arXiv, 10 juin 2016. arXiv.org, \url{https://arxiv.org/abs/1606.03498}. \label{OpenAI}
    \item Keras Documentation. (n.d.). LearningRateScheduler. Retrieved April 14, 2024, from \url{https://keras.io/api/callbacks/learning_rate_scheduler/}. \label{LR_S}
    \item Keras Documentation. (n.d.). ReduceLROnPlateau. Retrieved April 14, 2024, from \url{https://keras.io/api/callbacks/reduce_lr_on_plateau/}.\label{reduce_lr}
    \item Wang, Zhengwei, et al. Generative Adversarial Networks in Computer Vision: A Survey and Taxonomy. arXiv:1906.01529, arXiv, 29 décembre 2020. arXiv.org, \url{https://arxiv.org/pdf/1906.01529.pdf}. \label{list_GAN}
    %technical details
    \item TensorFlow Documentation. (n.d.). Retrieved April 14, 2024, from \url{https://www.tensorflow.org/}. \label{tf}
    \item Keras Documentation. (n.d.). Retrieved April 14, 2024, from \url{https://keras.io/}. \label{keras}
    \item PyPI. (n.d.). uproot. Retrieved April 14, 2024, from \url{https://pypi.org/project/uproot/}. \label{uproot}
    %Kolmogorov-Smirnov
    \item Massey, Frank J. « The Kolmogorov-Smirnov Test for Goodness of Fit ». Journal of the American Statistical Association, vol. 46, no 253, 1951, p. 68‑78. JSTOR, \url{https://doi.org/10.2307/2280095}. \label{KS test}
    \item Cuemath. (n.d.). Z-Test Calculator. Retrieved April 14, 2024, from \url{https://www.cuemath.com/data/z-test/} \label{Z test}
    %Param and nonparam
    \item « Difference between Parametric and Non-Parametric Methods ». GeeksforGeeks, 8 février 2020,\\
    \url{https://www.geeksforgeeks.org/difference-between-parametric-and-non-parametric-methods/}.
    %Monte Carlo
    \item Rummukainen, K. (n.d.). Monte Carlo simulations in physics. Department of Physical Sciences, University of Oulu, \url{https://www.mv.helsinki.fi/home/rummukai/lectures/montecarlo_oulu/lectures/mc_notes1.pdf}. \label{Monte Carlo}
    %Morphing
    \item Verkerke, W. (n.d.). Introduction to Morphing. Nikhef, \url{https://indico.cern.ch/event/507948/contributions/2028505/attachments/1262169/1866169/atlas-hcomb-morphwshop-intro-v1.pdf}. \label{morphing}
    %ABCD method
    \item Introduction – Background estimation with the ABCD method. \url{https://cms-opendata-workshop.github.io/workshop-lesson-abcd-method/01-introduction/index.html}. Consulté le 14 avril 2024. 
    \item CERN. (2018, October 18). ABCD Method Guide. Retrieved from \url{https://twiki.cern.ch/twiki/pub/Main/ABCDMethod/ABCDGuide_draft18Oct18.pdf}. \label{abcd}
    %cGAN approach
    \item ATLAS Collaboration. « Search for Higgs boson decays into a $Z$ boson and a light hadronically decaying resonance using 13 TeV $pp$ collision data from the ATLAS detector ». Physical Review Letters, vol. 125, no 22, novembre 2020, p. 221802. arXiv.org, \url{https://doi.org/10.1103/PhysRevLett.125.221802}. \label{atlas z}
    %POWHEG
    \item Homepage of the POWHEG BOX. \url{https://powhegbox.mib.infn.it/}. Consulté le 14 avril 2024.
    \label{POWHEG}
    %
    %Physics
    \item https://atlas.cern/updates/briefing/new-milestone-di-Higgs-search \label{di-higgs decay 20 times}
    \item Tumasyan, A., et al. « A Portrait of the Higgs Boson by the CMS Experiment Ten Years after the Discovery ». Nature, vol. 607, no 7917, juillet 2022, p. 60‑68. www.nature.com, \url{https://doi.org/10.1038/s41586-022-04892-x}.
    \label{Nature}
    \item Davies, C. T. H., et al. « High-Precision Lattice QCD Confronts Experiment ». Physical Review Letters, vol. 92, no 2, January 2004, p. 022001. arXiv.org, \url{https://doi.org/10.1103/PhysRevLett.92.022001}. \label{QCD lattice}


    
    %Generative AI
    %NF
    \item Kobyzev, Ivan, et al. « Normalizing Flows: An Introduction and Review of Current Methods ». IEEE Transactions on Pattern Analysis and Machine Intelligence, vol. 43, no 11, novembre 2021, p. 3964‑79. arXiv.org, \url{https://doi.org/10.1109/TPAMI.2020.2992934}. \label{NF}
    %VAE
    \item Pinheiro Cinelli, Lucas; et al. (2021). "Variational Autoencoder". Variational Methods for Machine Learning with Applications to Deep Networks. Springer. pp. 111–149. \url{doi:10.1007/978-3-030-70679-1_5} \label{VAE}
    %WCGAN
    \item Arjovsky, Martin, et al. Wasserstein GAN. arXiv:1701.07875, arXiv, 6 décembre 2017. arXiv.org, \url{https://arxiv.org/abs/1701.07875}. \label{WCGAN}
    %GAN + NF for speech enhacement
    \item Strauss, Martin, et al. « SEFGAN: Harvesting the Power of Normalizing Flows and GANs for Efficient High-Quality Speech Enhancement ». 2023 IEEE Workshop on Applications of Signal Processing to Audio and Acoustics (WASPAA), 2023, p. 1‑5. arXiv.org, \url{https://doi.org/10.1109/WASPAA58266.2023.10248144}. \label{speech}
    %Comparison between 3
    \item Bond-Taylor, Sam, et al. « Deep Generative Modelling: A Comparative Review of VAEs, GANs, Normalizing Flows, Energy-Based and Autoregressive Models ». IEEE Transactions on Pattern Analysis and Machine Intelligence, vol. 44, no 11, novembre 2022, p. 7327‑47. arXiv.org, \url{https://doi.org/10.1109/TPAMI.2021.3116668}. \label{GAN NF VAE}
    %GAN vs NF
    \item Liu, Tianci, et Jeffrey Regier. An Empirical Comparison of GANs and Normalizing Flows for Density Estimation. arXiv:2006.10175, arXiv, 14 décembre 2021. arXiv.org, \url{https://doi.org/10.48550/arXiv.2006.10175}. \label{GAN vs NF}
    
\end{enumerate}

\end{document}
